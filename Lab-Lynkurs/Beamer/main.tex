\documentclass[aspectratio=169]{beamer}
\usepackage[utf8]{inputenc}
\usepackage[norsk]{babel}
\usepackage{graphicx}
\usepackage{hyperref}
\usepackage{booktabs}
\usepackage{amsmath}
\usepackage{listings}
\usepackage{xcolor}

\usetheme{Madrid}
\usecolortheme{beaver}

\definecolor{codegreen}{rgb}{0,0.6,0}
\definecolor{codegray}{rgb}{0.5,0.5,0.5}
\definecolor{codepurple}{rgb}{0.58,0,0.82}
\definecolor{backcolour}{rgb}{0.95,0.95,0.92}

\lstdefinestyle{mystyle}{
    backgroundcolor=\color{backcolour},   
    commentstyle=\color{codegreen},
    keywordstyle=\color{magenta},
    numberstyle=\tiny\color{codegray},
    stringstyle=\color{codepurple},
    basicstyle=\ttfamily\footnotesize,
    breakatwhitespace=false,         
    breaklines=true,                 
    captionpos=b,                    
    keepspaces=true,                 
    numbers=left,                    
    numbersep=5pt,                  
    showspaces=false,                
    showstringspaces=false,
    showtabs=false,                  
    tabsize=2,
    literate={æ}{{\ae}}1 {ø}{{\o}}1 {å}{{\aa}}1 {Æ}{{\AE}}1 {Ø}{{\O}}1 {Å}{{\AA}}1
}
\lstset{style=mystyle}

\title{Lab-Lynkurs: Introduksjon til AI og Python}
\subtitle{ELMED219-2026: Kunstig intelligens og beregningsorientert medisin}
\author{ELMED219}
\date{Vår 2026}

\begin{document}

\frame{\titlepage}

\begin{frame}{Oversikt}
    \tableofcontents
\end{frame}

\section{Introduksjon til Kurset}
\begin{frame}{Velkommen til ELMED219}
    \begin{columns}
        \column{0.5\textwidth}
        \textbf{Mål for kurset:}
        \begin{itemize}
            \item Forstå grunnleggende AI-konsepter.
            \item Anvende maskinlæring på medisinske data.
            \item Bli kjent med Python som verktøy.
            \item Reflektere over etikk og 'trustworthy AI'.
        \end{itemize}
        
        \column{0.5\textwidth}
        \textbf{Struktur:}
        \begin{itemize}
            \item \textbf{Lab 0:} ML med PyCaret (Automl).
            \item \textbf{Lab 1:} Nettverksanalyse (PSN).
            \item \textbf{Lab 2:} Deep Learning (Bilder/EEG).
            \item \textbf{Lab 3:} Generativ AI (LLM).
            \item \textbf{Prosjekt:} Gruppearbeid.
        \end{itemize}
    \end{columns}
\end{frame}

\section{Python for Medisinere}
\begin{frame}{Hvorfor Python?}
    \begin{columns}
        \column{0.6\textwidth}
        Python har blitt de-facto standarden innen AI og Data Science.
        \begin{itemize}
            \item \textbf{Lesbarhet:} Syntaksen ligner på engelsk pseudo-kode.
            \item \textbf{Økosystem:} Enorme biblioteker for alt fra statistikk til bildeanalyse.
            \item \textbf{Community:} Stor støtte og mange eksempler på nettet (Stack Overflow, ChatGPT).
        \end{itemize}
        \vspace{0.5cm}
        \textit{"Python is executable pseudocode."}
        
        \column{0.4\textwidth}
        \begin{block}{Prompt}
            \scriptsize\textit{"En vennlig, stilisert pythonslange som slynger seg rundt et medisinsk kors, 3d render, søt, ren bakgrunn, høy kvalitet, medisinsk blått og hvitt fargeskjema."}
        \end{block}
        \centering
        \includegraphics[width=0.6\textwidth]{images/python_medisin.pdf}
    \end{columns}
\end{frame}

\begin{frame}{Python vs. Andre språk}
    \begin{table}[]
        \centering
        \small
        \begin{tabular}{@{}lll@{}}
            \toprule
            \textbf{Språk} & \textbf{Fordeler} & \textbf{Ulemper} \\ \midrule
            \textbf{Python} & Enkelt, AI-standard, fleksibelt & Tregere enn C++, GIL \\
            \textbf{R} & Fantastisk for statistikk & Mindre støtte for Deep Learning \\
            \textbf{MATLAB} & Industri-standard ingeniørfag & Dyrt, proprietært \\
            \textbf{C++} & Ekstremt raskt & Høy læringskurve, komplekst \\ \bottomrule
        \end{tabular}
    \end{table}
    \vspace{0.5cm}
    I ELMED219 bruker vi Python fordi bibliotekene (PyTorch, Scikit-learn) er best utviklet her.
\end{frame}

\section{Jupyter Notebooks}
\begin{frame}{Jupyter Notebook Miljøet}
    \begin{columns}
        \column{0.5\textwidth}
        En \textbf{Notebook} (.ipynb) kombinerer:
        \begin{enumerate}
            \item \textbf{Kjørbar kode} (Python celler).
            \item \textbf{Rik tekst} (Markdown, LaTeX formler).
            \item \textbf{Visualiseringer} (Plot direkte i dokumentet).
        \end{enumerate}
        Dette fremmer \textit{reproduserbar forskning}.
        
        \column{0.5\textwidth}
        \begin{block}{Prompt}
            \scriptsize\textit{"Futuristisk kodegrensesnitt med visualisering av medisinske data, holografisk, blått og hvitt tema, Python-kodesnutter som svever i luften, rent og moderne."}
        \end{block}
        \centering
        \includegraphics[width=0.6\textwidth]{images/coding_interface.pdf}
    \end{columns}
\end{frame}

\begin{frame}[fragile]{Markdown og LaTeX i Jupyter}
    Vi kan skrive formler direkte i tekst-celler ved å bruke \$:
    \vspace{0.5cm}
    
    \textbf{Eksempel:}
    \begin{verbatim}
    Massemiddelverdien er gitt ved $\mu = \frac{1}{n} \sum x_i$
    \end{verbatim}
    
    \textbf{Resultat:}
    Massemiddelverdien er gitt ved $\mu = \frac{1}{n} \sum x_i$
    
    \vspace{0.5cm}
    Dette er nyttig for å dokumentere matematiske metoder i lab-rapportene.
\end{frame}

\section{Viktige Biblioteker}
\begin{frame}{NumPy: Numerisk Python}
    Grunnlaget for all vitenskapelig beregning i Python.
    \begin{itemize}
        \item Introduserer \texttt{ndarray} (n-dimensjonale arrays).
        \item Mye raskere enn vanlige Python-lister pga. C-implementasjon og vektorisering.
    \end{itemize}
    
    \vspace{0.5cm}
    \begin{block}{Matematisk operasjon}
        Hvis $A$ og $B$ er matriser, vil \texttt{A * B} i NumPy ofte være elementvis multiplikasjon, mens \texttt{A @ B} er matrisemultiplikasjon.
    \end{block}
\end{frame}

\begin{frame}[fragile]{Pandas: Databehandling}
    Hovedverktøyet for tabulære data (som Excel, men kraftigere).
    
    \begin{columns}
        \column{0.5\textwidth}
        \begin{lstlisting}[language=Python]
import pandas as pd

# Read data
df = pd.read_csv('pasientdata.csv')

# Inspect first rows
print(df.head())

# Statistics
print(df.describe())
        \end{lstlisting}
        
        \column{0.5\textwidth}
        \textbf{Viktige begreper:}
        \begin{itemize}
            \item \texttt{DataFrame}: Hele tabellen.
            \item \texttt{Series}: En kolonne.
            \item \texttt{.loc[]}: Hente data ved navn.
            \item \texttt{.iloc[]}: Hente data ved indeks (posisjon).
        \end{itemize}
    \end{columns}
\end{frame}

\begin{frame}{Matplotlib \& Seaborn: Visualisering}
    \begin{itemize}
        \item \textbf{Matplotlib:} "Bestefaren" til plotting. Kraftig, men kan være omstendelig.
        \item \textbf{Seaborn:} Bygger på Matplotlib. Vakrere standard-design, enklere for statistiske plott.
    \end{itemize}
    
    \begin{center}
        \textit{En god visualisering kan erstatte tusen tall.}
    \end{center}
    Vi skal bruke disse til å visualisere forvirringsmatriser, treningskurver og feature importance.
\end{frame}

\section{Notebook Gjennomgang}
\begin{frame}{Oversikt over `lynkurs-ai-python.ipynb`}
    Notebooken er delt inn i følgende seksjoner:
    \begin{enumerate}
        \item \textbf{Python Basics:} Variabler, lister, løkker, funksjoner.
        \item \textbf{Numpy:} Vektorer og matriser.
        \item \textbf{Pandas:} Laste og utforske et "Heart Disease" datasett.
        \item \textbf{Visualisering:} Plotting av histogrammer og korrelasjoner.
        \item \textbf{Scikit-learn intro:} En enkel lineær regresjonsmodell.
    \end{enumerate}
\end{frame}

\begin{frame}[fragile]{Tips for Lab-arbeid}
    \begin{itemize}
        \item \textbf{Kjør cellene i rekkefølge!} Variabler må defineres før de brukes.
        \item \textbf{Tab-completion:} Trykk \texttt{Tab} for å se tilgjengelige funksjoner.
        \item \textbf{Hjelp:} Bruk \texttt{?} bak en funksjon for dokumentasjon.
        \begin{lstlisting}[language=Python]
pd.read_csv?
        \end{lstlisting}
        \item \textbf{Restart Kernel:} Hvis ting henger seg opp, start kjernen på nytt (Kernel -> Restart).
    \end{itemize}
\end{frame}

\begin{frame}{Veien Videre}
    Når dere er ferdige med lynkurset, er dere klare for \textbf{Lab 0}, hvor vi skal begynne med ekte maskinlæring ved hjelp av PyCaret.
    
    \vspace{1cm}
    \centering
    \textbf{Lykke til med kodingen!}
\end{frame}

\end{document}










