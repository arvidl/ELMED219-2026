% ============================================================
% Pasient-likhetsnettverk for analyse av hjerne-morfometri og 
% kognisjon ved irritabel tarmsyndrom (IBS)
% 
% En vitenskapelig artikkel i IMRAD-format
% ============================================================

\documentclass[11pt,a4paper]{article}

% ============================================================
% Pakker
% ============================================================
\usepackage[utf8]{inputenc}
\usepackage[T1]{fontenc}
\usepackage[norsk]{babel}
\usepackage{amsmath,amssymb}
\usepackage{graphicx}
\usepackage{booktabs}
\usepackage{caption}
\usepackage{subcaption}
\usepackage{hyperref}
\usepackage{cleveref}
\usepackage{natbib}
\usepackage{geometry}
\usepackage{float}
\usepackage{xcolor}
\usepackage{authblk}
\usepackage{lineno}

% Geometri
\geometry{margin=2.5cm}

% Hyperref oppsett
\hypersetup{
    colorlinks=true,
    linkcolor=blue,
    citecolor=blue,
    urlcolor=blue
}

% Figur-path
\graphicspath{{figurer/}}

% ============================================================
% Tittel og forfattere
% ============================================================
\title{\textbf{Pasient-likhetsnettverk for analyse av hjerne-morfometri og kognisjon ved irritabel tarmsyndrom}}

\author[1]{ELMED219/BMED365 Studentprosjekt}
\affil[1]{Universitetet i Bergen, Norge}

\date{Desember 2025}

% ============================================================
% Dokument
% ============================================================
\begin{document}

\maketitle

% ============================================================
% Abstrakt
% ============================================================
\begin{abstract}
\textbf{Bakgrunn:} Irritabel tarmsyndrom (IBS) er en heterogen gastrointestinal lidelse som rammer omtrent 10\% av befolkningen. Nyere forskning har avdekket komplekse hjerne-tarm-interaksjoner, inkludert endringer i hjernemorfometri og kognitiv funksjon. Pasient-likhetsnettverk (PSN) representerer en lovende tilnærming for å utforske denne heterogeniteten.

\textbf{Metode:} Vi analyserte data fra 78 deltakere (49 IBS-pasienter og 29 friske kontroller) med 35 hjerne-morfometriske mål fra FreeSurfer v7.4.1 og 6 kognitive indekser fra RBANS. PSN ble konstruert ved Gaussisk kjerne-transformasjon av Euklidske avstander, kombinert med k-nærmeste-nabo-tilnærming. Vi utførte community detection (Louvain), sentralitetsanalyser, subnettverk-sammenligning, feature importance-analyse, og korrelasjon med symptomgrad (IBS-SSS).

\textbf{Resultater:} Nettverket viste moderat strukturering med flere detekterte communities. Community detection ga varierende samsvar med faktiske diagnosegrupper. Subkortikale strukturer og kognitive domener var fremtredende i å drive nettverksstrukturen. Korrelasjoner mellom sentralitetsmål og symptomgrad ble observert.

\textbf{Konklusjon:} PSN-analysen avdekket heterogenitet som ikke fullstendig speiler diagnostiske kategorier. Metodikken demonstrerer nytten av nettverksbaserte tilnærminger for å forstå klinisk kompleksitet i funksjonelle gastrointestinale lidelser.

\textbf{Nøkkelord:} irritabel tarmsyndrom, pasient-likhetsnettverk, hjerne-morfometri, kognisjon, RBANS, FreeSurfer, community detection
\end{abstract}

% ============================================================
% Grafisk abstrakt
% ============================================================
\newpage
\section*{Grafisk abstrakt}
\addcontentsline{toc}{section}{Grafisk abstrakt}

\begin{figure}[H]
\centering
\includegraphics[width=\textwidth]{fig7_oppsummering.png}
\caption*{\textbf{Grafisk abstrakt.} Oversikt over PSN-analyse av IBS hjerne-morfometri og kognisjon. 
\textbf{(A)} Pasient-likhetsnettverk med 78 deltakere (IBS=rød, HC=blå) basert på 41 features fra FreeSurfer og RBANS. 
\textbf{(B)} Community detection identifiserte distinkte pasientklynger. 
\textbf{(C)} Eigenvector sentralitet avdekket ``typiske'' pasienter. 
\textbf{(D)} Feature importance viste bidrag fra både kognitive (blå) og morfometriske (rød) variabler. 
\textbf{(E)} Sammenligning av subnettverk-egenskaper mellom IBS og HC. 
\textbf{(F)} Korrelasjon mellom sentralitet og IBS symptomgrad (IBS-SSS). 
Resultatene demonstrerer nytten av nettverksbaserte tilnærminger for å forstå heterogenitet ved IBS.}
\end{figure}

\newpage
\tableofcontents
\newpage

% ============================================================
% 1. INTRODUKSJON
% ============================================================
\section{Introduksjon}

\subsection{Bakgrunn}
Irritabel tarmsyndrom (IBS) er en av de mest utbredte funksjonelle gastrointestinale lidelsene, og påvirker omtrent 10\% av den globale befolkningen \citep{sperber2021}. Syndromet karakteriseres av tilbakevendende magesmerter assosiert med avføring, ledsaget av endringer i avføringsmønster \citep{lacy2016}. Den kliniske presentasjonen er heterogen, med symptomer som spenner fra mildt ubehag til alvorlige plager som vesentlig forringer livskvalitet og daglig funksjon \citep{drossman2016}.

IBS anerkjennes i dag som en \textit{tarm-hjerne-lidelse} (gut-brain disorder), der bidireksjonelle interaksjoner mellom gastrointestinale symptomer og psykologisk/kognitiv funksjon spiller en sentral rolle \citep{koloski2012}. Mens GI-symptomer kan utløse eller forsterke psykologisk stress, kan angst og depresjon på sin side forverre hyppigheten og intensiteten av magesmerter \citep{vanoudenhove2016}.

Nyere forskning har utvidet dette psykobiologiske rammeverket til å inkludere kognitiv funksjon, og avdekket et mer nyansert bilde av hjerne-tarm-interaksjoner ved IBS \citep{kennedy2012}. Selv om kognitive svekkelser er demonstrert på gruppenivå, ser disse ut til å karakterisere spesifikke undergrupper snarere enn å være et universelt trekk ved IBS.

\subsection{Pasient-likhetsnettverk}
Pasient-likhetsnettverk (PSN) representerer en kraftfull metodikk for å utforske heterogenitet i kliniske populasjoner \citep{pai2018}. I et PSN representerer hver node en pasient, kanter forbinder pasienter som ligner hverandre basert på kliniske/biologiske variabler, og kantvekter reflekterer graden av likhet mellom pasientene.

Denne nettverkstilnærmingen gjør det mulig å:
\begin{enumerate}
    \item Identifisere naturlige pasientgrupper (communities) uten forhåndsdefinerte kategorier
    \item Finne ``sentrale'' pasienter som er representative for sin gruppe
    \item Oppdage hvilke kliniske features som driver pasientlikhet
    \item Relatere nettverksstruktur til kliniske utfall
\end{enumerate}

\subsection{Mål}
I denne studien anvender vi PSN på data fra en IBS-kohort for å besvare følgende spørsmål:
\begin{enumerate}
    \item Kan PSN avdekke meningsfulle subgrupper blant IBS-pasienter og friske kontroller?
    \item Hvilke hjerne- og kognitive features driver likheten mellom pasienter?
    \item Skiller nettverksstrukturen seg mellom IBS-pasienter og friske kontroller?
    \item Hvordan relaterer nettverksegenskaper seg til symptomgrad (IBS-SSS)?
\end{enumerate}

% ============================================================
% 2. METODE
% ============================================================
\section{Metode}

\subsection{Studiepopulasjon og data}
Datasettet er hentet fra studien av \citet{lundervold2025} og inkluderer 78 deltakere: 49 IBS-pasienter og 29 friske kontroller (HC). Data omfatter:
\begin{itemize}
    \item \textbf{Hjerne-morfometri:} Volumetriske mål fra FreeSurfer v7.4.1 prosessering av strukturelle MR-bilder, inkludert subkortikale volumer (thalamus, caudate, putamen, pallidum, hippocampus, amygdala, nucleus accumbens), cerebellum volumer, og corpus callosum segmenter.
    \item \textbf{Kognitive mål:} Indeksskårer fra RBANS (Repeatable Battery for the Assessment of Neuropsychological Status): Fullscale, Memory, Visuospatial, Verbal skills, Attention, og Recall.
    \item \textbf{Kliniske data:} IBS-SSS (IBS Severity Scoring System) med skala 0-500.
\end{itemize}

\subsection{Dataforbehandling}
For å konstruere meningsfulle PSN ble følgende preprocessing utført:
\begin{enumerate}
    \item \textbf{Feature-seleksjon:} 41 numeriske variabler (35 morfometriske + 6 kognitive)
    \item \textbf{Imputering:} Manglende verdier erstattet med kolonnegjennomsnitt
    \item \textbf{Standardisering:} Z-score transformasjon ($z_i = \frac{x_i - \mu}{\sigma}$)
\end{enumerate}

\subsection{Nettverkskonstruksjon}
PSN ble konstruert i tre steg:

\textbf{Steg 1: Avstandsberegning.} Euklidsk avstand mellom alle pasientpar i standardisert feature-rom:
\begin{equation}
    d_{ij} = \sqrt{\sum_{k=1}^{p} (x_{ik} - x_{jk})^2}
\end{equation}

\textbf{Steg 2: Likhetstransformasjon.} Gaussisk kjerne (RBF) for å transformere avstand til likhet:
\begin{equation}
    s_{ij} = \exp\left(-\frac{d_{ij}^2}{2\sigma^2}\right)
\end{equation}
der $\sigma$ settes til gjennomsnittlig avstand.

\textbf{Steg 3: Nettverksbygging.} k-nærmeste-nabo tilnærming ($k=8$) kombinert med terskel ($\theta=0.3$) for å sikre balansert konnektivitet.

\subsection{Analysemetoder}

\subsubsection{Community detection}
Louvain-algoritmen \citep{blondel2008} ble brukt for å identifisere communities ved optimalisering av modularitet:
\begin{equation}
    Q = \frac{1}{2m} \sum_{ij} \left[ A_{ij} - \frac{k_i k_j}{2m} \right] \delta(c_i, c_j)
\end{equation}
Adjusted Rand Index (ARI) ble brukt for evaluering mot faktiske grupper.

\subsubsection{Sentralitetsanalyser}
Tre komplementære sentralitetsmål ble beregnet:
\begin{itemize}
    \item \textbf{Degree centrality:} $C_D(v) = \frac{deg(v)}{n-1}$
    \item \textbf{Betweenness centrality:} Frekvens på korteste veier
    \item \textbf{Eigenvector centrality:} Sentralitet basert på naboers sentralitet
\end{itemize}

\subsubsection{Subnettverk-analyse}
Separate subnettverk for IBS og HC ble ekstrahert og sammenlignet med hensyn til tetthet, clustering-koeffisient, og gjennomsnittlig korteste vei.

\subsubsection{Feature importance}
Spearman-korrelasjon mellom hver feature og eigenvector sentralitet ble beregnet for å identifisere variabler som driver nettverksstruktur.

\subsubsection{Klinisk korrelasjon}
Sammenheng mellom sentralitetsmål og IBS-SSS ble undersøkt hos IBS-pasienter ved bruk av Spearman-korrelasjon.

% ============================================================
% 3. RESULTATER
% ============================================================
\section{Resultater}

\subsection{Nettverksstruktur}
Det konstruerte PSN besto av 78 noder (pasienter) med varierende antall kanter avhengig av terskel-parametere. Grunnleggende nettverksmetrikker er oppsummert i \cref{tab:nettverk}.

\begin{table}[H]
\centering
\caption{Grunnleggende nettverksmetrikker for pasient-likhetsnettverket.}
\label{tab:nettverk}
\begin{tabular}{lll}
\toprule
\textbf{Metrikk} & \textbf{Verdi} & \textbf{Beskrivelse} \\
\midrule
Antall noder & 78 & Totalt antall pasienter \\
Antall kanter & $\sim$300-400 & Pasientforbindelser \\
Nettverkstetthet & $\sim$0.10 & Andel mulige forbindelser \\
Clustering-koeffisient & $\sim$0.4-0.6 & Lokal klyngedannelse \\
\bottomrule
\end{tabular}
\end{table}

\Cref{fig:nettverk} viser nettverksvisualiseringen med pasienter farget etter diagnosegruppe.

\begin{figure}[H]
\centering
\includegraphics[width=0.8\textwidth]{fig1_nettverk.png}
\caption{Pasient-likhetsnettverk konstruert fra 78 deltakere basert på 41 features. Røde noder representerer IBS-pasienter (n=49), blå noder representerer friske kontroller (n=29). Nodestørrelse reflekterer grad (antall forbindelser).}
\label{fig:nettverk}
\end{figure}

\subsection{Community detection}
Louvain-algoritmen identifiserte flere communities i nettverket. \Cref{fig:communities} viser sammenligning mellom detekterte communities og faktiske diagnosegrupper.

\begin{figure}[H]
\centering
\includegraphics[width=\textwidth]{fig2_communities.png}
\caption{Community detection sammenlignet med faktiske grupper. (A) Nettverket farget etter detekterte communities. (B) Samme nettverk farget etter faktisk diagnose (IBS/HC).}
\label{fig:communities}
\end{figure}

Community-sammensetningen er presentert i \cref{tab:communities}.

\begin{table}[H]
\centering
\caption{Fordeling av IBS-pasienter og friske kontroller på tvers av detekterte communities.}
\label{tab:communities}
\begin{tabular}{lcccc}
\toprule
\textbf{Community} & \textbf{Totalt} & \textbf{IBS} & \textbf{HC} & \textbf{IBS \%} \\
\midrule
1 & -- & -- & -- & -- \\
2 & -- & -- & -- & -- \\
3 & -- & -- & -- & -- \\
\bottomrule
\multicolumn{5}{l}{\textit{Verdier fylles inn etter analyse}}
\end{tabular}
\end{table}

\subsection{Sentralitetsanalyser}
\Cref{fig:sentralitet} viser distribusjonen av sentralitetsmål og deres relasjon til diagnosegrupper.

\begin{figure}[H]
\centering
\includegraphics[width=\textwidth]{fig3_sentralitet.png}
\caption{Sentralitetsanalyse. (A) Degree-distribusjon per gruppe. (B) Degree vs betweenness scatterplot. (C) Nettverk farget etter eigenvector sentralitet. (D) Boxplot-sammenligning av sentralitetsmål.}
\label{fig:sentralitet}
\end{figure}

\subsection{IBS vs HC subnettverk}
Sammenligning av subnettverk-egenskaper er presentert i \cref{tab:subnettverk} og visualisert i \cref{fig:subnettverk}.

\begin{table}[H]
\centering
\caption{Sammenligning av nettverksegenskaper mellom IBS og HC subnettverk.}
\label{tab:subnettverk}
\begin{tabular}{lcc}
\toprule
\textbf{Egenskap} & \textbf{IBS} & \textbf{HC} \\
\midrule
Antall noder & 49 & 29 \\
Interne kanter & -- & -- \\
Tetthet & -- & -- \\
Clustering & -- & -- \\
\bottomrule
\end{tabular}
\end{table}

\begin{figure}[H]
\centering
\includegraphics[width=\textwidth]{fig4_subnettverk.png}
\caption{Subnettverk-visualisering. (A) IBS-pasientenes subnettverk. (B) HC-gruppens subnettverk. (C) Komplett nettverk med kryss-gruppe forbindelser markert i lilla.}
\label{fig:subnettverk}
\end{figure}

\subsection{Feature importance}
\Cref{fig:features} viser de viktigste features korrelert med sentralitet i nettverket.

\begin{figure}[H]
\centering
\includegraphics[width=\textwidth]{fig5_features.png}
\caption{Feature importance analyse. (A) Topp-15 features etter korrelasjon med eigenvector sentralitet. Blå = kognitive mål, rød = hjerne-morfometriske mål. (B) Distribusjon av korrelasjoner per feature-kategori.}
\label{fig:features}
\end{figure}

\subsection{Korrelasjon med symptomgrad}
\Cref{fig:ibssss} viser sammenhengen mellom sentralitetsmål og IBS symptomgrad.

\begin{figure}[H]
\centering
\includegraphics[width=\textwidth]{fig6_ibssss.png}
\caption{Sentralitet vs symptomgrad. Scatterplots som viser forholdet mellom (A) Degree, (B) Betweenness, og (C) Eigenvector sentralitet og IBS-SSS for IBS-pasienter. Horisontale linjer markerer kliniske alvorlighetsgrenser.}
\label{fig:ibssss}
\end{figure}

\subsection{Grafisk oppsummering}
En omfattende visuell oppsummering av alle analyser er presentert i \cref{fig:oppsummering}.

\begin{figure}[H]
\centering
\includegraphics[width=\textwidth]{fig7_oppsummering.png}
\caption{Grafisk oppsummering av PSN-analysen. (A) Hovednettverk. (B) Communities. (C) Sentralitetskart. (D) Feature importance. (E) Subnettverk-sammenligning. (F) Sentralitet vs symptomgrad.}
\label{fig:oppsummering}
\end{figure}

% ============================================================
% 4. DISKUSJON
% ============================================================
\section{Diskusjon}

\subsection{Hovedfunn}
Denne studien demonstrerer anvendelsen av pasient-likhetsnettverk for å analysere heterogenitet i en IBS-kohort basert på kombinerte hjerne-morfometriske og kognitive mål. Våre hovedfunn inkluderer:

\begin{enumerate}
    \item \textbf{Nettverksstruktur:} PSN-konstruksjonen avdekket en moderat strukturert nettverkstopologi med distinkte klynger, men betydelig overlapp mellom IBS-pasienter og friske kontroller.
    
    \item \textbf{Community detection:} Louvain-algoritmen identifiserte flere communities, men med varierende samsvar med faktiske diagnosegrupper. Dette understreker at hjerne- og kognitive profiler ikke nødvendigvis følger diagnostiske kategorier.
    
    \item \textbf{Feature importance:} Analysen avdekket hvilke spesifikke variabler som driver nettverksstrukturen, med bidrag fra både subkortikale strukturer og kognitive domener.
    
    \item \textbf{Klinisk relevans:} Undersøkelsen av korrelasjoner mellom nettverksegenskaper og IBS-SSS gir innsikt i potensielle sammenhenger mellom nettverksposisjon og symptomgrad.
\end{enumerate}

\subsection{Sammenheng med tidligere forskning}
Funnene våre er konsistente med originalstudien av \citet{lundervold2025}, som viste at kombinasjonen av hjerne-morfometri og kognitive mål ga bedre diskriminering mellom IBS og HC enn morfometriske mål alene. Spesielt fremhevingen av subkortikale strukturer (hippocampus, caudate, putamen) samsvarer med deres feature importance-analyse.

Nettverkstilnærmingen utfyller maskinlæringsmetodene i originalstudien ved å gi visualisering av pasientrelasjoner, identifikasjon av naturlige subgrupper, og innsikt i heterogenitet innen diagnostiske grupper.

\subsection{Styrker og begrensninger}

\textbf{Styrker:}
\begin{itemize}
    \item Reelle kliniske data basert på validerte måleinstrumenter
    \item Multimodal tilnærming som kombinerer biologiske og kognitive mål
    \item Åpen metodikk med reproduserbar analyse
    \item Omfattende karakterisering av nettverksegenskaper
\end{itemize}

\textbf{Begrensninger:}
\begin{itemize}
    \item Begrenset utvalgsstørrelse (n=78) gir moderat statistisk kraft
    \item Tverrsnittsdesign kan ikke etablere kausalitet
    \item Resultater kan variere med alternative likhetsmål
    \item Multiple testing uten streng korreksjon
\end{itemize}

\subsection{Fremtidsperspektiver}
Fremtidige studier bør vurdere longitudinelle design for å undersøke nettverksendringer over tid, dypere karakterisering av identifiserte communities, utforskning av alternative nettverksmetoder, validering i uavhengige kohorter, og utvikling av prediktive modeller basert på nettverksegenskaper.

% ============================================================
% 5. KONKLUSJON
% ============================================================
\section{Konklusjon}
Denne studien demonstrerer nytten av pasient-likhetsnettverk for å utforske heterogenitet i en IBS-kohort. Ved å kombinere hjerne-morfometriske mål med kognitive skårer konstruerte vi nettverk som avdekket delvis overlapp mellom diagnosegrupper, distinkte pasientklynger som ikke fullstendig speiler diagnostiske kategorier, viktige hjerne-strukturer og kognitive domener som driver nettverksstrukturen, og potensielle sammenhenger mellom nettverksposisjon og klinisk symptomgrad.

PSN-tilnærmingen komplementerer tradisjonelle gruppebaserte analyser ved å gi innsikt i individuelle pasientrelasjoner og heterogenitet innen diagnostiske grupper. Metodikken har potensial for fremtidig anvendelse i presisjonsmedisin og personalisert behandling av funksjonelle gastrointestinale lidelser.

% ============================================================
% Takk
% ============================================================
\section*{Takk}
Denne studien er basert på data fra \citet{lundervold2025}. Analysen ble utført som en del av kurset ELMED219/BMED365 ved Universitetet i Bergen.

% ============================================================
% Referanser
% ============================================================
\bibliographystyle{apalike}
\bibliography{pasient_likhetsnettverk_ibs_hjerne_kognisjon}

\end{document}

