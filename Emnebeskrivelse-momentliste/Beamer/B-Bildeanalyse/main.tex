% =====================================================================
% ELMED219: Medisinsk bildeanalyse
% Beamer-presentasjon - Momentliste B01-B06
% =====================================================================
\documentclass[aspectratio=169, 10pt]{beamer}

% =====================================================================
% PAKKER
% =====================================================================
\usepackage[utf8]{inputenc}
\usepackage[T1]{fontenc}
\usepackage[norsk]{babel}
\usepackage{graphicx}
\usepackage{tikz}
\usetikzlibrary{shapes.geometric, arrows, positioning, calc}
\usepackage{booktabs}
\usepackage{amsmath}
\usepackage{fontawesome5}
\usepackage{hyperref}
\hypersetup{colorlinks=true, linkcolor=blue, urlcolor=blue}

% =====================================================================
% TEMA OG FARGER
% =====================================================================
\usetheme{Madrid}
\usecolortheme{beaver}

% =====================================================================
% TITTELINFO
% =====================================================================
\title{Medisinsk Bildeanalyse}
\subtitle{ELMED219: Momentliste B01--B06}
\author{ELMED219}
\date{Vår 2026}

% =====================================================================
% DOKUMENT
% =====================================================================
\begin{document}

% Tittelside
\begin{frame}
    \titlepage
\end{frame}

% Innholdsfortegnelse
\begin{frame}{Oversikt}
    \tableofcontents
\end{frame}

% =====================================================================
% SEKSJON: MRI-grunnleggende
% =====================================================================
\section{MRI-grunnleggende}

\begin{frame}{B01: Forklare grunnleggende MRI-prinsipper på et overordnet nivå}
    \textbf{Magnetic Resonance Imaging (MRI):}
    
    \vspace{0.3cm}
    \textbf{Fysiske prinsipper (forenklet):}
    \begin{enumerate}
        \item \textbf{Magnetisering:} Sterk magnetfelt justerer hydrogenatomer (protoner)
        \item \textbf{RF-puls:} Radiofrekvens-puls ``vipper'' protonene
        \item \textbf{Relaksasjon:} Protoner returnerer til likevekt, sender ut signal
        \item \textbf{Deteksjon:} Signal måles og rekonstrueres til bilde
    \end{enumerate}
    
    \vspace{0.3cm}
    \textbf{Hvorfor MRI er viktig i medisin:}
    \begin{columns}[T]
        \begin{column}{0.48\textwidth}
            \begin{itemize}
                \item Ingen ioniserende stråling
                \item Utmerket bløtvevskontrast
                \item Mange ulike sekvenser/kontraster
            \end{itemize}
        \end{column}
        \begin{column}{0.48\textwidth}
            \begin{itemize}
                \item Kan avbilde funksjon (fMRI)
                \item Gullstandard for hjerne, ledd, rygg
                \item Multiparametrisk -- rik informasjon
            \end{itemize}
        \end{column}
    \end{columns}
    
    \begin{block}{Relevans for AI}
        MRI gir enorme datamengder -- ideell for AI-basert analyse, segmentering, og mønstergjenkjenning.
    \end{block}
\end{frame}

\begin{frame}{B02: Kjenne til ulike MR-sekvenser (T1, T1+Gd, T2, FLAIR)}
    \textbf{Vanlige MR-sekvenser og deres kontrast:}

    \vspace{0.2cm}
    \begin{center}
        \footnotesize
        \begin{tabular}{lp{4cm}p{4.5cm}}
            \toprule
            \textbf{Sekvens} & \textbf{Karakteristikker} & \textbf{Klinisk bruk} \\
            \midrule
            \textbf{T1-vektet} & CSF mørk, fett lyst, god anatomi & Anatomi \\
            \addlinespace
            \textbf{T1 + Gd} & Kontrast viser blod-hjerne-barriere-brudd & Aktiv tumor, metastaser \\
            \addlinespace
            \textbf{T2-vektet} & CSF lys, patologi ofte lys & Ødem, inflammasjon, væske \\
            \addlinespace
            \textbf{FLAIR} & T2-lik, men CSF mørk & Lesjoner nær ventrikler, MS-plakk \\
            \addlinespace
            \textbf{DWI} & Diffusjon av vannmolekyler & Akutt hjerneslag (innen minutter!) \\
            \bottomrule
        \end{tabular}
    \end{center}

    \vspace{0.2cm}
    \begin{alertblock}{\footnotesize For AI-analyse}
        \footnotesize Multiparametrisk MRI (flere sekvenser) gir komplementær informasjon som AI kan utnytte for bedre klassifisering.
    \end{alertblock}
\end{frame}

% =====================================================================
% SEKSJON: Segmentering
% =====================================================================
\section{Segmentering}

\begin{frame}{B03: Beskrive segmentering i medisinsk bildeanalyse}
    \textbf{Segmentering = identifisere og avgrense strukturer i bilde}
    
    \vspace{0.3cm}
    \textbf{Oppgave:}
    \begin{itemize}
        \item Tilordne hver piksel/voksel til en \textbf{klasse} (f.eks. tumor, normalvev, bakgrunn)
        \item Produserer et \textbf{segmenteringskart} (maske)
    \end{itemize}
    
    \vspace{0.3cm}
    \textbf{Typer segmentering:}
    \begin{columns}[T]
        \begin{column}{0.48\textwidth}
            \textbf{Semantisk segmentering:}
            \begin{itemize}
                \item Klassifiser hver piksel
                \item Alle tumorer = én klasse
            \end{itemize}
        \end{column}
        \begin{column}{0.48\textwidth}
            \textbf{Instans-segmentering:}
            \begin{itemize}
                \item Skiller individuelle objekter
                \item Tumor 1, Tumor 2, ...
            \end{itemize}
        \end{column}
    \end{columns}
    
    \vspace{0.3cm}
    \textbf{AI-metoder for segmentering:}
    \begin{itemize}
        \item \textbf{U-Net:} Klassisk encoder-decoder arkitektur (medisinsk standard)
        \item \textbf{\href{https://github.com/MIC-DKFZ/nnUNet}{nnU-Net}:} Selvkonfigurerende, state-of-the-art
        \item \textbf{Segment Anything (SAM):} Generell segmentering
    \end{itemize}
\end{frame}

\begin{frame}{B04: Kjenne til BraTS-utfordringen for hjernesvulstsegmentering}
    \textbf{\href{https://www.synapse.org/brats}{BraTS} = Brain Tumor Segmentation Challenge}

    \vspace{0.2cm}
    \begin{columns}[T]
        \begin{column}{0.48\textwidth}
            \textbf{Hva er BraTS?}
            \begin{itemize}
                \item Årlig internasjonal konkurranse siden 2012
                \item Benchmark for AI-segmentering av hjernesvulster
                \item Multiparametrisk MRI (T1, T1+Gd, T2, FLAIR)
            \end{itemize}

            \vspace{0.2cm}
            \textbf{Segmenteringsoppgave:}
            \begin{itemize}
                \item \textbf{Whole tumor (WT):} All tumorvev
                \item \textbf{Tumor core (TC):} Aktiv + nekrotisk
                \item \textbf{Enhancing tumor (ET):} Kontrastoppladende
            \end{itemize}
        \end{column}
        \begin{column}{0.48\textwidth}
            \textbf{Betydning:}
            \begin{itemize}
                \item Standardisert datasett for sammenligning
                \item Driver fremskritt innen medisinsk AI
                \item Grunnlag for klinisk translasjon
            \end{itemize}

            \vspace{0.2cm}
            \begin{block}{\footnotesize Relevans for Teamprosjekt}
                \footnotesize
                BraTS er direkte relevant for glioblastom-prosjektet -- studenter kan referere til BraTS-arkitekturer.
            \end{block}
        \end{column}
    \end{columns}
\end{frame}

% =====================================================================
% SEKSJON: Kvantitativ avbildning
% =====================================================================
\section{Kvantitativ avbildning}

\begin{frame}{B05: Beskrive radiomic features og kvantitativ avbildning}
    \textbf{\href{https://pyradiomics.readthedocs.io/}{Radiomics} = kvantifisering av bildeegenskaper}
    
    \vspace{0.3cm}
    \textbf{Hva er radiomic features?}
    \begin{itemize}
        \item Matematisk beskrivelse av tekstur, form, intensitet i bildet
        \item Hundrevis av features fra én region of interest (ROI)
        \item ``Dype'' egenskaper ikke synlig for øyet
    \end{itemize}
    
    \vspace{0.3cm}
    \textbf{Feature-kategorier:}
    \begin{columns}[T]
        \begin{column}{0.32\textwidth}
            \textbf{Intensitet:}
            \begin{itemize}
                \item Gjennomsnitt
                \item Standardavvik
                \item Histogram-features
            \end{itemize}
        \end{column}
        \begin{column}{0.32\textwidth}
            \textbf{Tekstur (GLCM):}
            \begin{itemize}
                \item Homogenitet
                \item Kontrast
                \item Entropi
            \end{itemize}
        \end{column}
        \begin{column}{0.32\textwidth}
            \textbf{Form:}
            \begin{itemize}
                \item Volum
                \item Overflate
                \item Sfærisitet
            \end{itemize}
        \end{column}
    \end{columns}
    
    \vspace{0.3cm}
    \begin{block}{Klinisk anvendelse}
        Radiomics kan predikere molekylære markører (IDH-status), prognose, og behandlingsrespons -- basert kun på standardbilder.
    \end{block}
\end{frame}

% =====================================================================
% SEKSJON: Verktøy
% =====================================================================
\section{Verktøy for medisinsk bildeanalyse}

\begin{frame}{B06: Kjenne til nnU-Net og MONAI som verktøy}
    \begin{columns}[T]
        \begin{column}{0.48\textwidth}
            \textbf{nnU-Net:}
            \begin{itemize}
                \item ``No-new-Net'' -- selvkonfigurerende
                \item Analyserer data og velger optimal konfigurasjon
                \item Preprocessing, arkitektur, trening
                \item State-of-the-art på mange benchmarks
                \item Krever kun data -- resten er automatisk
            \end{itemize}
            
            \vspace{0.3cm}
            \textbf{Styrke:}
            \begin{itemize}
                \item Minimal manuell tuning
                \item Reproduserbar baseline
            \end{itemize}
        \end{column}
        \begin{column}{0.48\textwidth}
            \textbf{\href{https://monai.io/}{MONAI}:}
            \begin{itemize}
                \item Medical Open Network for AI
                \item PyTorch-basert rammeverk
                \item Spesialisert for medisinsk avbildning
                \item Transformasjoner, nettverk, metrikker
                \item Aktiv utvikler-community
            \end{itemize}
            
            \vspace{0.3cm}
            \textbf{Styrke:}
            \begin{itemize}
                \item Fleksibilitet for forskning
                \item Ferdigbygde komponenter
            \end{itemize}
        \end{column}
    \end{columns}
    
    \vspace{0.2cm}
    \begin{block}{\footnotesize Andre verktøy}
        \footnotesize \textbf{3D Slicer:} Visualisering og analyse | \textbf{ITK/SimpleITK:} Bildebehandling | \textbf{Nipype:} Neuro-pipelines
    \end{block}
\end{frame}

% =====================================================================
% OPPSUMMERING
% =====================================================================
\section*{Oppsummering}

\begin{frame}{Oppsummering: Medisinsk Bildeanalyse}
    \textbf{Nøkkelpunkter:}
    \begin{itemize}
        \item \textbf{B01:} MRI-prinsipper -- magnetisering, RF-puls, relaksasjon, deteksjon
        \item \textbf{B02:} MR-sekvenser -- T1, T1+Gd, T2, FLAIR, DWI
        \item \textbf{B03:} Segmentering -- klassifiser hver piksel, U-Net, nnU-Net
        \item \textbf{B04:} BraTS -- benchmark for hjernesvulstsegmentering
        \item \textbf{B05:} Radiomics -- kvantitative features for prediksjon
        \item \textbf{B06:} Verktøy -- nnU-Net (automatisk), MONAI (fleksibelt)
    \end{itemize}

    \vspace{0.2cm}
    \begin{block}{\footnotesize Teamprosjekt-relevans}
        \footnotesize Medisinsk bildeanalyse er kjernen i teamprosjektet om glioblastom. Bruk kunnskapen om MRI, segmentering og radiomics.
    \end{block}

    \begin{block}{\footnotesize Lab 2}
        \footnotesize Praktisk erfaring med CNN for bildeanalyse, inkludert medisinsk anvendelse (MR-demens).
    \end{block}
\end{frame}

\end{document}

