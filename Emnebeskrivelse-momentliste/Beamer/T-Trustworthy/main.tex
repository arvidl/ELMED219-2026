% =====================================================================
% ELMED219: Trustworthy AI og robusthet
% Beamer-presentasjon - Momentliste T01-T07
% =====================================================================
\documentclass[aspectratio=169, 10pt]{beamer}

% =====================================================================
% PAKKER
% =====================================================================
\usepackage[utf8]{inputenc}
\usepackage[T1]{fontenc}
\usepackage[norsk]{babel}
\usepackage{graphicx}
\usepackage{tikz}
\usetikzlibrary{shapes.geometric, arrows, positioning, calc}
\usepackage{booktabs}
\usepackage{amsmath}
\usepackage{fontawesome5}
\usepackage{hyperref}
\hypersetup{
    colorlinks=true,
    linkcolor=blue,
    urlcolor=blue,
    citecolor=blue
}

% =====================================================================
% TEMA OG FARGER
% =====================================================================
\usetheme{Madrid}
\usecolortheme{beaver}

% =====================================================================
% TITTELINFO
% =====================================================================
\title{Trustworthy AI og Robusthet}
\subtitle{ELMED219: Momentliste T01--T07}
\author{ELMED219}
\date{Vår 2026}

% =====================================================================
% DOKUMENT
% =====================================================================
\begin{document}

% Tittelside
\begin{frame}
    \titlepage
\end{frame}

% Innholdsfortegnelse
\begin{frame}{Oversikt}
    \begin{enumerate}
        \item \textbf{Trustworthy AI}
        \begin{itemize}
            \item T01: Definere trustworthy AI iht. EU-retningslinjer
        \end{itemize}
        \item \textbf{Robusthet og usikkerhet}
        \begin{itemize}
            \item T02: Forklare konseptet robusthet i ML/AI
            \item T03: Beskrive datadrift og dens konsekvenser
            \item T04: Forklare epistemisk vs. aleatorisk usikkerhet
        \end{itemize}
        \item \textbf{Menneske-maskin samspill}
        \begin{itemize}
            \item T05: Beskrive human-in-the-loop (HITL) systemer
            \item T06: Diskutere viktigheten av kontinuerlig monitorering
        \end{itemize}
        \item \textbf{Sikkerhetstrusler}
        \begin{itemize}
            \item T07: Kjenne til adversarial attacks
        \end{itemize}
    \end{enumerate}
\end{frame}

% =====================================================================
% SEKSJON: Trustworthy AI
% =====================================================================
\section{Trustworthy AI}

\begin{frame}{T01: Definere trustworthy AI iht. EU-retningslinjer}
    \textbf{\href{https://digital-strategy.ec.europa.eu/en/policies/expert-group-ai}{EU High-Level Expert Group on AI} (2019):}
    
    \begin{block}{Definisjon}
        \textbf{Trustworthy AI} er AI som er lovlig, etisk og robust -- både teknisk og sosialt.
    \end{block}
    
    \vspace{0.2cm}
    \textbf{Syv nøkkelkrav:}
    \begin{enumerate}
        \item \textbf{Menneskers handlingsrom og tilsyn} -- Human agency \& oversight
        \item \textbf{Teknisk robusthet og sikkerhet} -- Technical robustness \& safety
        \item \textbf{Personvern og dataforvaltning} -- Privacy \& data governance
        \item \textbf{Transparens} -- Transparency
        \item \textbf{Mangfold, ikke-diskriminering, rettferdighet} -- Diversity, non-discrimination, fairness
        \item \textbf{Sosial og miljømessig velferd} -- Societal \& environmental well-being
        \item \textbf{Ansvarlighet} -- Accountability
    \end{enumerate}
    
    \vspace{0.1cm}
    {\small
    \begin{alertblock}{Medisinsk relevans}
        Høyrisiko AI-systemer i helse må oppfylle disse kravene under \href{https://artificialintelligenceact.eu/}{EU AI Act}.
    \end{alertblock}
    }
\end{frame}

% =====================================================================
% SEKSJON: Robusthet
% =====================================================================
\section{Robusthet og usikkerhet}

\begin{frame}{T02: Forklare konseptet robusthet i ML/AI}
    \textbf{Robusthet = modellens evne til å prestere pålitelig under variasjon}
    
    \vspace{0.2cm}
    \textbf{En robust modell:}
    \begin{itemize}
        \item Gir \textbf{konsistente resultater} på lignende input
        \item \textbf{Degraderer gradvis} (\href{https://en.wikipedia.org/wiki/Graceful_degradation}{ikke katastrofalt}) ved støy
        \item Generaliserer godt til nye, usette data
    \end{itemize}
    
    \vspace{0.2cm}
    \textbf{Typer robusthet:}
    \begin{columns}[T]
        \begin{column}{0.48\textwidth}
            \begin{itemize}
                \item \textbf{Støyrobusthet:} Toleranse for tilfeldig støy
                \item \textbf{Distribusjonell robusthet:} Endringer i datafordeling
            \end{itemize}
        \end{column}
        \begin{column}{0.48\textwidth}
            \begin{itemize}
                \item \textbf{Adversarial robusthet:} Motstand mot bevisste angrep
                \item \textbf{Temporal robusthet:} Stabilitet over tid
            \end{itemize}
        \end{column}
    \end{columns}
    
    \vspace{0.15cm}
    {\small
    \begin{block}{I medisin}
        En robust medisinsk AI gir pålitelige prediksjoner uavhengig av variasjon i bildekvalitet, pasientpopulasjon, eller utstyr.
    \end{block}
    }
\end{frame}

\begin{frame}{T03: Beskrive datadrift og dens konsekvenser}
    \textbf{\href{https://en.wikipedia.org/wiki/Data_drift}{Datadrift} (distributional shift):}
    \begin{itemize}
        \item Forskjell mellom \textbf{treningsdata} og \textbf{produksjonsdata}
        \item Modellen møter data som ikke ligner det den er trent på
    \end{itemize}
    
    \vspace{0.2cm}
    \textbf{Typer drift:}
    \begin{enumerate}
        \item \textbf{\href{https://en.wikipedia.org/wiki/Covariate_shift}{Covariate shift}:} Input-fordeling endres (f.eks. ny pasientdemografi)
        \item \textbf{\href{https://en.wikipedia.org/wiki/Label_shift}{Label shift}:} Forekomst av klasser endres (f.eks. pandemier)
        \item \textbf{\href{https://en.wikipedia.org/wiki/Concept_drift}{Concept drift}:} Sammenhengen mellom input og output endres
    \end{enumerate}
    
    \vspace{0.2cm}
    \textbf{Eksempler i medisin:}
    \begin{itemize}
        \item Modell trent på data fra USA brukes i Norge
        \item Ny MR-skanner gir andre bildekarakteristikker
        \item COVID endret innleggelsesmønstre
    \end{itemize}
    
    {\small
    \begin{alertblock}{Konsekvens}
        Modellen kan \textbf{\href{https://en.wikipedia.org/wiki/Silent_failure}{feile stille}} -- gir prediksjoner med høy konfidens som er feil!
    \end{alertblock}
    }
\end{frame}

\begin{frame}{T04: Forklare forskjellen mellom epistemisk og aleatorisk usikkerhet}
    \begin{columns}[T]
        \begin{column}{0.48\textwidth}
            \textbf{\href{https://en.wikipedia.org/wiki/Uncertainty_quantification\#Epistemic_uncertainty}{Epistemisk usikkerhet}:}
            \begin{itemize}
                \item Usikkerhet pga. \textbf{manglende kunnskap}
                \item ``Vi vet ikke nok ennå''
                \item \textbf{Kan reduseres} med mer data
                \item Modellen er usikker på områder med lite treningsdata
            \end{itemize}
            
            \vspace{0.3cm}
            \textbf{Eksempel:}
            \begin{itemize}
                \item Sjelden sykdom med få treningseksempler
                \item Usikkerhet fordi modellen mangler erfaring
            \end{itemize}
        \end{column}
        \begin{column}{0.48\textwidth}
            \textbf{\href{https://en.wikipedia.org/wiki/Uncertainty_quantification\#Aleatoric_uncertainty}{Aleatorisk usikkerhet}:}
            \begin{itemize}
                \item \textbf{Iboende} tilfeldig variasjon i data
                \item ``Verden er usikker''
                \item \textbf{Kan ikke reduseres} med mer data
                \item Støy i målinger, naturlig variasjon
            \end{itemize}
            
            \vspace{0.3cm}
            \textbf{Eksempel:}
            \begin{itemize}
                \item To pasienter med identiske features har ulike utfall
                \item Usikkerhet fordi utfall er genuint usikkert
            \end{itemize}
        \end{column}
    \end{columns}
    
    \vspace{0.5cm}
    \begin{block}{Praktisk betydning}
        Epistemisk usikkerhet signaliserer når modellen bør ``si ifra'' at den er usikker, og mennesker bør overta.
    \end{block}
\end{frame}

% =====================================================================
% SEKSJON: Menneske-maskin samspill
% =====================================================================
\section{Menneske-maskin samspill}

\begin{frame}{T05: Beskrive human-in-the-loop (HITL) systemer}
    \textbf{\href{https://en.wikipedia.org/wiki/Human-in-the-loop}{Human-in-the-loop} (HITL):}
    \vspace{-0.1cm}
    \begin{itemize}
        \item Mennesker er \textbf{integrert} i AI-systemets beslutningsprosess
        \item AI gir anbefalinger, mennesker tar endelige beslutninger
    \end{itemize}
    
    \vspace{-0.05cm}
    \textbf{Tre hovedvarianter:}
    \vspace{-0.1cm}
    \begin{enumerate}
        \item \textbf{\href{https://en.wikipedia.org/wiki/Human-in-the-loop}{Human-in-the-loop}:} Menneske involveres i hver beslutning
        \item \textbf{\href{https://en.wikipedia.org/wiki/Human-on-the-loop}{Human-on-the-loop}:} Menneske overvåker og kan overstyre
        \item \textbf{\href{https://en.wikipedia.org/wiki/Automation\#Levels_of_automation}{Human-out-of-the-loop}:} Full autonomi (ikke anbefalt i medisin)
    \end{enumerate}
    
    \vspace{-0.05cm}
    \textbf{Fordeler med HITL i medisin:}
    \vspace{-0.1cm}
    \begin{itemize}
        \item Kombinerer AI-effektivitet med menneskelig ekspertise
        \item Fanger opp AI-feil før de får konsekvenser
        \item Opprettholder klinisk ansvar, bygger tillit gradvis
    \end{itemize}
    
    {\footnotesize
    \begin{alertblock}{\href{https://artificialintelligenceact.eu/}{EU AI Act}}
        Høyrisiko AI-systemer \textbf{krever} effektiv menneskelig tilsyn.
    \end{alertblock}
    }
\end{frame}

\begin{frame}{T06: Diskutere viktigheten av kontinuerlig monitorering}
    \textbf{Hvorfor \href{https://en.wikipedia.org/wiki/Continuous_monitoring}{kontinuerlig monitorering}?}
    \begin{itemize}
        \item ML-modeller \textbf{degraderer over tid} (model drift)
        \item Verden endrer seg, data endrer seg
        \item Feil i produksjon kan ha alvorlige konsekvenser
    \end{itemize}
    
    \vspace{0.2cm}
    \textbf{Hva bør overvåkes?}
    \begin{columns}[T]
        \begin{column}{0.48\textwidth}
            \textbf{Ytelsesmetrikker:}
            \begin{itemize}
                \item Accuracy, precision, recall over tid
                \item Kalibrering av konfidensscorer
                \item Sammenligning med baseline
            \end{itemize}
        \end{column}
        \begin{column}{0.48\textwidth}
            \textbf{Datakarakteristikker:}
            \begin{itemize}
                \item Input-fordeling (drift-deteksjon)
                \item Andel out-of-distribution input
                \item Feature-statistikk
            \end{itemize}
        \end{column}
    \end{columns}
    
    \vspace{0.2cm}
    \textbf{Tiltak ved problemer:}
    \begin{itemize}
        \item \textbf{Re-trening} med ferske data -- oppdater modellen på nyere, representative data
        \item \textbf{Varsling og eskalering} -- automatisk varsling ved ytelsesfall eller drift
        \item \textbf{Fallback} -- bytt til enklere modell eller menneskelig vurdering ved usikkerhet
    \end{itemize}
\end{frame}

% =====================================================================
% SEKSJON: Sikkerhetstrusler
% =====================================================================
\section{Sikkerhetstrusler}

\begin{frame}{T07: Kjenne til adversarial attacks}
    \textbf{\href{https://en.wikipedia.org/wiki/Adversarial_machine_learning}{Adversarial attacks} = bevisst manipulering av AI-input}
    
    \begin{itemize}
        \item Små, tilsynelatende usynlige endringer som lurer modellen
        \item Kan få en korrekt klassifikasjon til å bli \textbf{fullstendig feil}
    \end{itemize}
    
    \vspace{0.2cm}
    \textbf{Eksempler:}
    \begin{columns}[T]
        \begin{column}{0.48\textwidth}
            \textbf{Bilder:}
            \begin{itemize}
                \item Piksel-perturbasjoner usynlige for mennesker
                \item ``Panda'' $\rightarrow$ ``Gibbon'' med 99\% konfidens
                \item \href{https://arxiv.org/abs/1707.08945}{Klistremerker som lurer selvkjørende biler}
            \end{itemize}
        \end{column}
        \begin{column}{0.48\textwidth}
            \textbf{Tekst (LLM):}
            \begin{itemize}
                \item \href{https://en.wikipedia.org/wiki/Prompt_injection}{Prompt injection}
                \item \href{https://en.wikipedia.org/wiki/Jailbreaking_(LLM)}{Jailbreaking}
                \item Omgå \href{https://en.wikipedia.org/wiki/Content_moderation}{sikkerhetsfiltre}
            \end{itemize}
        \end{column}
    \end{columns}
    
    \vspace{0.15cm}
    {\small
    \begin{alertblock}{Medisinsk risiko og forsvar}
        \textbf{Risiko:} En angriper kunne manipulere et røntgenbilde slik at AI overser patologi, eller skaper falske funn. \\
        \textbf{Forsvar:} Adversarial training, input-validering, ensemble-metoder, robusthetstesting.
    \end{alertblock}
    }
\end{frame}

% =====================================================================
% OPPSUMMERING
% =====================================================================
\section*{Oppsummering}

\begin{frame}{Oppsummering: Trustworthy AI og Robusthet}
    \textbf{Nøkkelpunkter:}
    \begin{itemize}
        \item \textbf{T01:} EU's 7 krav til trustworthy AI
        \item \textbf{T02:} Robusthet -- konsistente resultater under variasjon
        \item \textbf{T03:} Distributional shift -- når data i praksis avviker fra trening
        \item \textbf{T04:} Epistemisk (kan reduseres) vs. aleatorisk (iboende) usikkerhet
        \item \textbf{T05:} HITL -- mennesker i beslutningssløyfen
        \item \textbf{T06:} Kontinuerlig monitorering -- fange drift og degradering
        \item \textbf{T07:} Adversarial attacks -- bevisst manipulering
    \end{itemize}
    
    \vspace{0.3cm}
    \begin{block}{Hovedbudskap}
        Trustworthy AI i medisin krever robuste modeller, menneskelig tilsyn, kontinuerlig monitorering og bevissthet om sikkerhetsrisikoer. Teknikk alene er ikke nok -- det kreves også organisatoriske og regulatoriske tiltak.
    \end{block}
\end{frame}

\end{document}









