% =====================================================================
% ELMED219: Praktiske ferdigheter
% Beamer-presentasjon - Momentliste F01-F10
% =====================================================================
\documentclass[aspectratio=169, 11pt]{beamer}

% =====================================================================
% PAKKER
% =====================================================================
\usepackage[utf8]{inputenc}
\usepackage[T1]{fontenc}
\usepackage[norsk]{babel}
\usepackage{graphicx}
\usepackage{tikz}
\usetikzlibrary{shapes.geometric, arrows, positioning, calc}
\usepackage{booktabs}
\usepackage{amsmath}
\usepackage{fontawesome5}
\usepackage{listings}

% Kodeformatering
\lstset{
    basicstyle=\ttfamily\small,
    breaklines=true,
    frame=single,
    backgroundcolor=\color{lightgray},
    keywordstyle=\color{uibblue},
    stringstyle=\color{uibred},
}

% =====================================================================
% TEMA OG FARGER
% =====================================================================
\usetheme{Madrid}
\usecolortheme{default}

% UiB-farger
\definecolor{uibblue}{RGB}{0, 61, 115}
\definecolor{uibred}{RGB}{175, 28, 44}
\definecolor{lightgray}{RGB}{245, 245, 245}

\setbeamercolor{palette primary}{bg=uibblue, fg=white}
\setbeamercolor{palette secondary}{bg=uibblue!80, fg=white}
\setbeamercolor{palette tertiary}{bg=uibblue!60, fg=white}
\setbeamercolor{palette quaternary}{bg=uibblue, fg=white}
\setbeamercolor{structure}{fg=uibblue}
\setbeamercolor{section in toc}{fg=uibblue}
\setbeamercolor{frametitle}{fg=uibblue, bg=lightgray}
\setbeamercolor{title}{fg=white, bg=uibblue}
\setbeamercolor{block title}{bg=uibblue, fg=white}
\setbeamercolor{block body}{bg=lightgray, fg=black}
\setbeamercolor{block title alerted}{bg=uibred, fg=white}
\setbeamercolor{block body alerted}{bg=uibred!10, fg=black}

% Fjern navigasjonssymboler
\setbeamertemplate{navigation symbols}{}

% Enkel footer med sidetall
\setbeamertemplate{footline}{
    \hfill\insertframenumber/\inserttotalframenumber\hspace{2mm}\vspace{2mm}
}

% =====================================================================
% TITTELINFO
% =====================================================================
\title[F: Praktiske ferdigheter]{Praktiske Ferdigheter}
\subtitle{Momentliste F01--F10}
\author{ELMED219 / BMED365}
\institute{Universitetet i Bergen}
\date{Våren 2026}

% =====================================================================
% DOKUMENT
% =====================================================================
\begin{document}

% Tittelside
\begin{frame}
    \titlepage
\end{frame}

% Innholdsfortegnelse
\begin{frame}{Oversikt}
    \tableofcontents
\end{frame}

% =====================================================================
% SEKSJON: Jupyter og Colab
% =====================================================================
\section{Jupyter Notebooks og Google Colab}

\begin{frame}{F01: Kjøre Jupyter Notebooks i Google Colab}
    \textbf{Hva er Jupyter Notebooks?}
    \begin{itemize}
        \item Interaktive dokumenter som kombinerer kode, tekst og visualiseringer
        \item Kjør Python-kode celle for celle
        \item Standard i datavitenskapelig arbeid
    \end{itemize}
    
    \vspace{0.3cm}
    \textbf{Google Colab:}
    \begin{itemize}
        \item Gratis sky-basert Jupyter-miljø fra Google
        \item Ingen installasjon -- kjører i nettleseren
        \item Gratis GPU-tilgang (viktig for dyplæring!)
        \item Integrert med Google Drive
    \end{itemize}
    
    \vspace{0.3cm}
    \textbf{Kom i gang:}
    \begin{enumerate}
        \item Gå til \texttt{colab.research.google.com}
        \item Logg inn med Google-konto
        \item ``New notebook'' eller åpne fra GitHub
    \end{enumerate}
    
    \begin{block}{Tips}
        Aktiver GPU: Runtime $\rightarrow$ Change runtime type $\rightarrow$ GPU
    \end{block}
\end{frame}

% =====================================================================
% SEKSJON: Python-grunnleggende
% =====================================================================
\section{Python-grunnleggende}

\begin{frame}[fragile]{F02: Bruke Python-variabler, lister og enkle funksjoner}
    \textbf{Variabler:}
    \begin{lstlisting}[language=Python]
alder = 45          # int
navn = "Pasient A"  # str
risiko = 0.73       # float
er_syk = True       # bool
    \end{lstlisting}
    
    \textbf{Lister:}
    \begin{lstlisting}[language=Python]
symptomer = ["hodepine", "kvalme", "tretthet"]
verdier = [1.2, 3.4, 5.6, 7.8]
symptomer.append("feber")  # Legg til element
    \end{lstlisting}
    
    \textbf{Funksjoner:}
    \begin{lstlisting}[language=Python]
def beregn_bmi(vekt, hoyde):
    return vekt / (hoyde ** 2)

bmi = beregn_bmi(70, 1.75)  # -> 22.9
    \end{lstlisting}
\end{frame}

\begin{frame}[fragile]{F03: Importere og bruke biblioteker (numpy, pandas, matplotlib)}
    \textbf{Import av biblioteker:}
    \begin{lstlisting}[language=Python]
import numpy as np
import pandas as pd
import matplotlib.pyplot as plt
    \end{lstlisting}
    
    \vspace{0.3cm}
    \begin{columns}[T]
        \begin{column}{0.48\textwidth}
            \textbf{NumPy:}
            \begin{itemize}
                \item Numeriske beregninger
                \item Arrays og matriser
                \item Matematiske funksjoner
            \end{itemize}
            \begin{lstlisting}[language=Python]
arr = np.array([1, 2, 3])
mean = np.mean(arr)
            \end{lstlisting}
        \end{column}
        \begin{column}{0.48\textwidth}
            \textbf{Pandas:}
            \begin{itemize}
                \item Datamanipulering
                \item DataFrames (tabeller)
                \item CSV, Excel I/O
            \end{itemize}
            \begin{lstlisting}[language=Python]
df = pd.read_csv("data.csv")
df.head()
            \end{lstlisting}
        \end{column}
    \end{columns}
\end{frame}

\begin{frame}[fragile]{F04: Lese og inspisere datasett med pandas}
    \textbf{Lese data:}
    \begin{lstlisting}[language=Python]
df = pd.read_csv("pasienter.csv")
    \end{lstlisting}
    
    \textbf{Inspisere data:}
    \begin{lstlisting}[language=Python]
df.head()        # Første 5 rader
df.info()        # Kolonnetyper, nullverdier
df.describe()    # Statistikk for numeriske kolonner
df.shape         # (antall rader, antall kolonner)
df.columns       # Kolonnenavn
    \end{lstlisting}
    
    \textbf{Filtrering og utvalg:}
    \begin{lstlisting}[language=Python]
# Velg kolonner
df[["alder", "diagnose"]]

# Filtrer rader
eldre = df[df["alder"] > 65]
    \end{lstlisting}
\end{frame}

% =====================================================================
% SEKSJON: Maskinlæring med scikit-learn
% =====================================================================
\section{Maskinlæring med scikit-learn}

\begin{frame}[fragile]{F05: Trene en enkel modell med scikit-learn}
    \textbf{Typisk ML-workflow:}
    \begin{lstlisting}[language=Python]
from sklearn.model_selection import train_test_split
from sklearn.tree import DecisionTreeClassifier
from sklearn.metrics import accuracy_score

# 1. Forbered data
X = df[["feature1", "feature2"]]
y = df["label"]

# 2. Del i trening og test
X_train, X_test, y_train, y_test = train_test_split(
    X, y, test_size=0.2, random_state=42)

# 3. Lag og tren modell
model = DecisionTreeClassifier()
model.fit(X_train, y_train)

# 4. Evaluer
y_pred = model.predict(X_test)
print(f"Accuracy: {accuracy_score(y_test, y_pred):.2f}")
    \end{lstlisting}
\end{frame}

\begin{frame}[fragile]{F06: Visualisere resultater med matplotlib}
    \textbf{Grunnleggende plotting:}
    \begin{lstlisting}[language=Python]
import matplotlib.pyplot as plt

# Linjediagram
plt.plot([1, 2, 3, 4], [10, 20, 25, 30])
plt.xlabel("Tid"); plt.ylabel("Verdi")
plt.title("Min figur")
plt.show()
    \end{lstlisting}
    
    \begin{columns}[T]
        \begin{column}{0.48\textwidth}
            \textbf{Histogram:}
            \begin{lstlisting}[language=Python]
plt.hist(df["alder"], bins=20)
plt.xlabel("Alder")
plt.show()
            \end{lstlisting}
        \end{column}
        \begin{column}{0.48\textwidth}
            \textbf{Scatter plot:}
            \begin{lstlisting}[language=Python]
plt.scatter(df["x"], df["y"])
plt.xlabel("X"); plt.ylabel("Y")
plt.show()
            \end{lstlisting}
        \end{column}
    \end{columns}
    
    \vspace{0.3cm}
    \begin{block}{Seaborn}
        \texttt{import seaborn as sns} -- Penere visualiseringer med enkel kode!
    \end{block}
\end{frame}

% =====================================================================
% SEKSJON: Nettverksanalyse
% =====================================================================
\section{Nettverksanalyse med NetworkX}

\begin{frame}[fragile]{F07: Bruke NetworkX for enkel nettverksanalyse}
    \textbf{Opprett og manipuler grafer:}
    \begin{lstlisting}[language=Python]
import networkx as nx

# Opprett graf
G = nx.Graph()
G.add_nodes_from(["P1", "P2", "P3", "P4"])
G.add_edges_from([("P1", "P2"), ("P2", "P3"), ("P1", "P3")])

# Grunnleggende analyse
print(f"Antall noder: {G.number_of_nodes()}")
print(f"Antall kanter: {G.number_of_edges()}")

# Sentralitet
deg_cent = nx.degree_centrality(G)
print(f"Degree centrality: {deg_cent}")

# Visualisering
nx.draw(G, with_labels=True, node_color="lightblue")
plt.show()
    \end{lstlisting}
\end{frame}

% =====================================================================
% SEKSJON: Dyplæring
% =====================================================================
\section{Dyplæring med PyTorch}

\begin{frame}[fragile]{F08: Bygge og trene en modell med PyTorch}
    \textbf{Enkel MLP i PyTorch:}
    \begin{lstlisting}[language=Python]
import torch
import torch.nn as nn

class SimpleMLP(nn.Module):
    def __init__(self, input_size, hidden_size, num_classes):
        super().__init__()
        self.fc1 = nn.Linear(input_size, hidden_size)
        self.relu = nn.ReLU()
        self.fc2 = nn.Linear(hidden_size, num_classes)
    
    def forward(self, x):
        out = self.fc1(x)
        out = self.relu(out)
        out = self.fc2(out)
        return out

model = SimpleMLP(784, 128, 10)  # MNIST-eksempel
    \end{lstlisting}
    
    \begin{block}{Lab 2}
        Full trening med loss, optimizer, og treningsløkke gjennomgås i Lab 2.
    \end{block}
\end{frame}

% =====================================================================
% SEKSJON: AI-verktøy
% =====================================================================
\section{AI-verktøy og LaTeX}

\begin{frame}{F09: Bruke AI-verktøy (ChatGPT, Gemini) som kodehjelp}
    \textbf{AI som programmeringspartner:}
    
    \vspace{0.3cm}
    \textbf{Nyttige bruksområder:}
    \begin{itemize}
        \item \faCode~\textbf{Forklare kode:} ``Hva gjør denne funksjonen?''
        \item \faBug~\textbf{Feilsøking:} ``Hvorfor får jeg denne feilmeldingen?''
        \item \faLightbulb~\textbf{Forslag:} ``Hvordan kan jeg gjøre dette mer effektivt?''
        \item \faBook~\textbf{Dokumentasjon:} ``Hvilke parametre tar denne funksjonen?''
        \item \faPuzzlePiece~\textbf{Generere kode:} ``Skriv en funksjon som...''
    \end{itemize}
    
    \vspace{0.3cm}
    \textbf{Tips for effektiv bruk:}
    \begin{enumerate}
        \item Vær \textbf{spesifikk} i spørsmålene
        \item Inkluder \textbf{kontekst} (feilmelding, kodeeksempel)
        \item \textbf{Verifiser} alltid AI-generert kode
        \item Bruk som \textbf{læringsverktøy}, ikke bare kopier
    \end{enumerate}
    
    \begin{alertblock}{Gemini i Colab}
        Gemini er integrert direkte i Google Colab -- bruk det!
    \end{alertblock}
\end{frame}

\begin{frame}[fragile]{F10: Skrive enkle dokumenter med LaTeX/Overleaf}
    \textbf{LaTeX = profesjonelt dokumentformat}
    
    \vspace{0.3cm}
    \textbf{Hvorfor LaTeX?}
    \begin{itemize}
        \item Standard i akademisk publisering
        \item Utmerket for formler og figurer
        \item Versjonskontroll (tekst-basert)
    \end{itemize}
    
    \vspace{0.3cm}
    \textbf{Grunnleggende struktur:}
    \begin{lstlisting}[language=TeX]
\documentclass{article}
\usepackage[utf8]{inputenc}

\title{Min rapport}
\author{Student}
\date{Januar 2026}

\begin{document}
\maketitle
\section{Introduksjon}
Tekst her...
\end{document}
    \end{lstlisting}
    
    \begin{block}{Overleaf}
        \texttt{overleaf.com} -- Online LaTeX-editor, ingen installasjon, samarbeid i sanntid
    \end{block}
\end{frame}

% =====================================================================
% OPPSUMMERING
% =====================================================================
\section*{Oppsummering}

\begin{frame}{Oppsummering: Praktiske ferdigheter}
    \textbf{Jupyter og Python:}
    \begin{itemize}
        \item F01--F04: Colab, variabler, lister, funksjoner, pandas
    \end{itemize}
    
    \textbf{Maskinlæring:}
    \begin{itemize}
        \item F05--F06: scikit-learn workflow, matplotlib visualisering
    \end{itemize}
    
    \textbf{Spesialisert:}
    \begin{itemize}
        \item F07: NetworkX for nettverksanalyse
        \item F08: PyTorch for dyplæring
    \end{itemize}
    
    \textbf{Verktøy:}
    \begin{itemize}
        \item F09: AI-assistenter som kodehjelp
        \item F10: LaTeX/Overleaf for akademisk skriving
    \end{itemize}
    
    \vspace{0.3cm}
    \begin{block}{Lynkurs og Labs}
        Alle disse ferdighetene praktiseres gjennom Lynkurset og Lab 0--3. Øving gir mestring!
    \end{block}
\end{frame}

\end{document}
