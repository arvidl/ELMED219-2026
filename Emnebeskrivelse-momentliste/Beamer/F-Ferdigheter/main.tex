% =====================================================================
% ELMED219: Praktiske ferdigheter
% Beamer-presentasjon - Momentliste F01-F10
% =====================================================================
\documentclass[aspectratio=169, 10pt]{beamer}

% =====================================================================
% PAKKER
% =====================================================================
\usepackage[utf8]{inputenc}
\usepackage[T1]{fontenc}
\usepackage[norsk]{babel}
\usepackage{graphicx}
\usepackage{tikz}
\usetikzlibrary{shapes.geometric, arrows, positioning, calc}
\usepackage{booktabs}
\usepackage{amsmath}
\usepackage{fontawesome5}
\usepackage{listings}
\usepackage{xcolor}
\usepackage{hyperref}
\hypersetup{colorlinks=true, linkcolor=blue, urlcolor=blue}

% Kodeformatering
\definecolor{codebg}{rgb}{0.95,0.95,0.92}
\lstset{
    basicstyle=\ttfamily\footnotesize,
    breaklines=true,
    frame=single,
    backgroundcolor=\color{codebg},
}

% =====================================================================
% TEMA OG FARGER
% =====================================================================
\usetheme{Madrid}
\usecolortheme{beaver}

% =====================================================================
% TITTELINFO
% =====================================================================
\title{Praktiske Ferdigheter}
\subtitle{ELMED219: Momentliste F01--F10}
\author{ELMED219}
\date{Vår 2026}

% =====================================================================
% DOKUMENT
% =====================================================================
\begin{document}

% Tittelside
\begin{frame}
    \titlepage
\end{frame}

% Innholdsfortegnelse
\begin{frame}{Oversikt}
    \begin{enumerate}
        \item \textbf{Jupyter Notebooks og Google Colab}
        (\href{https://jupyter.org/}{Jupyter}, \href{https://colab.research.google.com/}{Colab})
        \item \textbf{Python-grunnleggende}
        (\href{https://numpy.org/}{NumPy}, \href{https://pandas.pydata.org/}{Pandas}, \href{https://matplotlib.org/}{Matplotlib})
        \item \textbf{Maskinlæring med scikit-learn}
        (\href{https://scikit-learn.org/}{scikit-learn})
        \item \textbf{Nettverksanalyse med NetworkX}
        (\href{https://networkx.org/}{NetworkX})
        \item \textbf{Dyplæring med PyTorch}
        (\href{https://pytorch.org/}{PyTorch})
        \item \textbf{AI-verktøy og LaTeX}
        (\href{https://chat.openai.com/}{ChatGPT}, \href{https://gemini.google.com/}{Gemini}, \href{https://claude.ai/}{Claude}, \href{https://www.overleaf.com/}{Overleaf})
    \end{enumerate}
\end{frame}

% =====================================================================
% SEKSJON: Jupyter og Colab
% =====================================================================
\section{Jupyter Notebooks og Google Colab}

\begin{frame}{F01: Kjøre Jupyter Notebooks i Google Colab}
    \textbf{Hva er Jupyter Notebooks?}
    \begin{itemize}
        \item Interaktive dokumenter som kombinerer kode, tekst og visualiseringer
        \item Kjør Python-kode celle for celle
        \item Standard i datavitenskapelig arbeid
    \end{itemize}
    
    \vspace{0.3cm}
    \textbf{Google Colab:}
    \begin{itemize}
        \item Gratis sky-basert Jupyter-miljø fra Google
        \item Ingen installasjon -- kjører i nettleseren
        \item Gratis GPU-tilgang (viktig for dyplæring!)
        \item Integrert med Google Drive
    \end{itemize}
    
    \vspace{0.3cm}
    \textbf{Kom i gang:}
    \begin{enumerate}
        \item Gå til \texttt{colab.research.google.com}
        \item Logg inn med Google-konto
        \item ``New notebook'' eller åpne fra GitHub
    \end{enumerate}
    
    \begin{block}{\footnotesize Tips}
        \footnotesize Aktiver GPU: Runtime $\rightarrow$ Change runtime type $\rightarrow$ GPU
    \end{block}
\end{frame}

% =====================================================================
% SEKSJON: Python-grunnleggende
% =====================================================================
\section{Python-grunnleggende}

\begin{frame}[fragile]{F02: Bruke Python-variabler, lister og enkle funksjoner}
    \textbf{Variabler:}
    \begin{lstlisting}[language=Python]
alder = 45          # int
navn = "Pasient A"  # str
risiko = 0.73       # float
er_syk = True       # bool
    \end{lstlisting}
    
    \textbf{Lister:}
    \begin{lstlisting}[language=Python]
symptomer = ["hodepine", "kvalme", "tretthet"]
verdier = [1.2, 3.4, 5.6, 7.8]
symptomer.append("feber")  # Legg til element
    \end{lstlisting}
    
    \textbf{Funksjoner:}
    \begin{lstlisting}[language=Python]
def beregn_bmi(vekt, hoyde):
    return vekt / (hoyde ** 2)

bmi = beregn_bmi(70, 1.75)  # -> 22.9
    \end{lstlisting}
\end{frame}

\begin{frame}[fragile]{F03: Importere og bruke biblioteker (numpy, pandas, matplotlib)}
    \textbf{Import av biblioteker:}
    \begin{lstlisting}[language=Python]
import numpy as np
import pandas as pd
import matplotlib.pyplot as plt
    \end{lstlisting}
    
    \vspace{0.3cm}
    \begin{columns}[T]
        \begin{column}{0.48\textwidth}
            \textbf{NumPy:}
            \begin{itemize}
                \item Numeriske beregninger
                \item Arrays og matriser
                \item Matematiske funksjoner
            \end{itemize}
            \begin{lstlisting}[language=Python]
arr = np.array([1, 2, 3])
mean = np.mean(arr)
            \end{lstlisting}
        \end{column}
        \begin{column}{0.48\textwidth}
            \textbf{Pandas:}
            \begin{itemize}
                \item Datamanipulering
                \item DataFrames (tabeller)
                \item CSV, Excel I/O
            \end{itemize}
            \begin{lstlisting}[language=Python]
df = pd.read_csv("data.csv")
df.head()
            \end{lstlisting}
        \end{column}
    \end{columns}
\end{frame}

\begin{frame}[fragile]{F04: Lese og inspisere datasett med pandas}
    \textbf{Lese data:}
    \begin{lstlisting}[language=Python]
df = pd.read_csv("pasienter.csv")
    \end{lstlisting}
    
    \textbf{Inspisere data:}
    \begin{lstlisting}[language=Python]
df.head()        # Første 5 rader
df.info()        # Kolonnetyper, nullverdier
df.describe()    # Statistikk for numeriske kolonner
df.shape         # (antall rader, antall kolonner)
df.columns       # Kolonnenavn
    \end{lstlisting}
    
    \textbf{Filtrering og utvalg:}
    \begin{lstlisting}[language=Python]
# Velg kolonner
df[["alder", "diagnose"]]

# Filtrer rader
eldre = df[df["alder"] > 65]
    \end{lstlisting}
\end{frame}

% =====================================================================
% SEKSJON: Maskinlæring med scikit-learn
% =====================================================================
\section{Maskinlæring med scikit-learn}

\begin{frame}[fragile]{F05: Trene en enkel modell med scikit-learn}
    \textbf{Typisk ML-workflow:}
    \begin{lstlisting}[language=Python]
from sklearn.model_selection import train_test_split
from sklearn.tree import DecisionTreeClassifier
from sklearn.metrics import accuracy_score

# 1. Forbered data
X = df[["feature1", "feature2"]]
y = df["label"]

# 2. Del i trening og test
X_train, X_test, y_train, y_test = train_test_split(
    X, y, test_size=0.2, random_state=42)

# 3. Lag og tren modell
model = DecisionTreeClassifier()
model.fit(X_train, y_train)

# 4. Evaluer
y_pred = model.predict(X_test)
print(f"Accuracy: {accuracy_score(y_test, y_pred):.2f}")
    \end{lstlisting}
\end{frame}

\begin{frame}[fragile]{F06: Visualisere resultater med matplotlib}
    \textbf{Grunnleggende plotting:}
    \begin{lstlisting}[language=Python]
import matplotlib.pyplot as plt

# Linjediagram
plt.plot([1, 2, 3, 4], [10, 20, 25, 30])
plt.xlabel("Tid"); plt.ylabel("Verdi")
plt.title("Min figur")
plt.show()
    \end{lstlisting}
    
    \begin{columns}[T]
        \begin{column}{0.48\textwidth}
            \textbf{Histogram:}
            \begin{lstlisting}[language=Python]
plt.hist(df["alder"], bins=20)
plt.xlabel("Alder")
plt.show()
            \end{lstlisting}
        \end{column}
        \begin{column}{0.48\textwidth}
            \textbf{Scatter plot:}
            \begin{lstlisting}[language=Python]
plt.scatter(df["x"], df["y"])
plt.xlabel("X"); plt.ylabel("Y")
plt.show()
            \end{lstlisting}
        \end{column}
    \end{columns}
    
    \vspace{0.2cm}
    \begin{block}{\footnotesize \href{https://seaborn.pydata.org/}{Seaborn}}
        \footnotesize \texttt{import seaborn as sns} -- Penere visualiseringer med enkel kode!
    \end{block}
\end{frame}

% =====================================================================
% SEKSJON: Nettverksanalyse
% =====================================================================
\section{Nettverksanalyse med NetworkX}

\begin{frame}[fragile]{F07: Bruke \href{https://networkx.org/}{NetworkX} for enkel nettverksanalyse}
    \textbf{Opprett og manipuler grafer:}
    \begin{lstlisting}[language=Python]
import networkx as nx

# Opprett graf
G = nx.Graph()
G.add_nodes_from(["P1", "P2", "P3", "P4"])
G.add_edges_from([("P1", "P2"), ("P2", "P3"), ("P1", "P3")])

# Grunnleggende analyse
print(f"Antall noder: {G.number_of_nodes()}")
print(f"Antall kanter: {G.number_of_edges()}")

# Sentralitet
deg_cent = nx.degree_centrality(G)
print(f"Degree centrality: {deg_cent}")

# Visualisering
nx.draw(G, with_labels=True, node_color="lightblue")
plt.show()
    \end{lstlisting}
\end{frame}

% =====================================================================
% SEKSJON: Dyplæring
% =====================================================================
\section{Dyplæring med PyTorch}

\begin{frame}[fragile]{F08: Bygge og trene en modell med \href{https://pytorch.org/}{PyTorch}}
    \textbf{Enkel MLP i PyTorch:}
    \begin{lstlisting}[language=Python,basicstyle=\ttfamily\scriptsize]
import torch
import torch.nn as nn

class SimpleMLP(nn.Module):
    def __init__(self, input_size, hidden_size, num_classes):
        super().__init__()
        self.fc1 = nn.Linear(input_size, hidden_size)
        self.relu = nn.ReLU()
        self.fc2 = nn.Linear(hidden_size, num_classes)

    def forward(self, x):
        out = self.fc1(x)
        out = self.relu(out)
        out = self.fc2(out)
        return out

model = SimpleMLP(784, 128, 10)  # MNIST-eksempel
    \end{lstlisting}

    \begin{block}{\footnotesize Lab 2}
        \footnotesize Full trening med loss, optimizer, og treningsløkke gjennomgås i Lab 2.
    \end{block}
\end{frame}

% =====================================================================
% SEKSJON: AI-verktøy
% =====================================================================
\section{AI-verktøy og LaTeX}

\begin{frame}{F09: Bruke AI-verktøy (\href{https://chat.openai.com/}{ChatGPT}, \href{https://gemini.google.com/}{Gemini}, \href{https://claude.ai/}{Claude}) som kodehjelp}
    \begin{columns}[T]
        \begin{column}{0.52\textwidth}
            \textbf{Nyttige bruksområder:}
            \begin{itemize}
                \item \faCode~\textbf{Forklare kode:} ``Hva gjør denne funksjonen?''
                \item \faBug~\textbf{Feilsøking:} ``Hvorfor får jeg feilmeldingen?''
                \item \faLightbulb~\textbf{Forslag:} ``Gjør dette mer effektivt?''
                \item \faPuzzlePiece~\textbf{Generere kode:} ``Skriv en funksjon...''
            \end{itemize}
        \end{column}
        \begin{column}{0.45\textwidth}
            \textbf{Tips for effektiv bruk:}
            \begin{enumerate}
                \item Vær \textbf{spesifikk} i spørsmålene
                \item Inkluder \textbf{kontekst}
                \item \textbf{Verifiser} alltid AI-kode
                \item Bruk som \textbf{læringsverktøy}
            \end{enumerate}
        \end{column}
    \end{columns}

    \vspace{0.1cm}
    \begin{block}{\footnotesize AI-drevne IDE-er (Integrated Development Environment)}
        \scriptsize
        \textbf{IDE} = Editor + kompilator + debugger + verktøy i ett.
        \textbf{AI-drevet IDE} = IDE med LLM-agenter som kan planlegge, skrive, teste og validere kode autonomt.\\
        \textit{Betydning for medisinsk/biomedisinsk forskning:} Senker terskelen for å utvikle analyseverktøy, akselererer prototyping av ML-modeller, og muliggjør raskere oversettelse fra forskning til klinikk.\\[3pt]
        \begin{tabular}{@{}ll@{}}
        \href{https://www.cursor.com/}{Cursor}, \href{https://windsurf.com/}{Windsurf} & Editor med agenter, plan mode \\
        \href{https://antigravity.dev/}{Google Antigravity} & Agent-first, multi-modell (Gemini 3) \\
        \href{https://replit.com/}{Replit Agent} & Sky-basert IDE med AI-agent \\
        \href{https://lovable.dev/}{Lovable} & Naturlig språk $\rightarrow$ full-stack app \\
        \href{https://github.com/features/copilot}{GitHub Copilot}, \href{https://www.jetbrains.com/ai/}{JetBrains AI} & AI-assistenter i eksisterende IDE-er
        \end{tabular}
    \end{block}
\end{frame}

\begin{frame}[fragile]{F10: Skrive dokumenter med \href{https://www.latex-project.org/}{LaTeX}/\href{https://www.overleaf.com/}{Overleaf}}
    \begin{columns}[T]
        \begin{column}{0.28\textwidth}
            \textbf{\href{http://www.bibtex.org/}{BibTeX} (.bib-fil):}
            \begin{lstlisting}[language=TeX,basicstyle=\ttfamily\tiny]
@article{smith2024,
  author  = {Smith, J.},
  title   = {AI in Medicine},
  journal = {Nature Med.},
  year    = {2024},
  volume  = {30},
  pages   = {123--130},
  doi     = {10.1038/s41591-
             024-01234-5}
}

@book{bishop2006,
  author    = {Bishop, C.},
  title     = {Pattern Recog.},
  publisher = {Springer},
  year      = {2006},
  isbn      = {978-0387310732}
}
            \end{lstlisting}
            \tiny I tekst: \texttt{\textbackslash cite\{smith2024\}}
        \end{column}
        \begin{column}{0.40\textwidth}
            \textbf{IMRAD-struktur:}
            \begin{lstlisting}[language=TeX,basicstyle=\ttfamily\tiny,literate={å}{{\aa}}1]
\documentclass{article}
\usepackage{graphicx,booktabs}
\begin{document}
\title{Tittel}
\author{Forfatter}
\maketitle
\begin{abstract} ... \end{abstract}
% Innledning
\section{Introduction}
Bakgrunn og mål \cite{smith2024}.
% Materiale og metoder
\section{Methods}
Datainnsamling og analyse.
% Resultater
\section{Results}
\begin{table}[h]
\begin{tabular}{lcc} \toprule
Modell & AUC & F1 \\ \midrule
CNN & 0.92 & 0.88 \\ \bottomrule
\end{tabular}
\caption{Resultater}\label{tab:1}
\end{table}
% Diskusjon
\section{Discussion}
Implikasjoner og begrensninger.
\bibliography{refs}
\end{document}
            \end{lstlisting}
        \end{column}
        \begin{column}{0.30\textwidth}
            \textbf{Tidsskrift m/LaTeX:}
            \tiny
            \begin{itemize}
                \item \href{https://journals.plos.org/plosmedicine/s/latex}{PLOS Medicine}
                \item \href{https://www.nature.com/nature/for-authors/formatting-guide}{Nature / Nat. Med.}
                \item \href{https://academic.oup.com/bioinformatics/pages/instructions_for_authors}{Bioinformatics}
                \item \href{https://www.thelancet.com/pb/assets/raw/Lancet/authors/tl-info-for-authors.pdf}{The Lancet}
                \item \href{https://www.bmj.com/about-bmj/resources-authors}{BMJ}
                \item \href{https://www.elsevier.com/researcher/author/policies-and-guidelines/latex-instructions}{Elsevier (mange)}
                \item \href{https://www.springer.com/gp/livingreviews/latex-templates}{Springer Nature}
                \item \href{https://www.mdpi.com/authors/latex}{MDPI (open access)}
                \item \href{https://ieee-dataport.org/sites/default/files/analysis/27/IEEE-Reference-Guide.pdf}{IEEE JBHI}
                \item \href{https://www.jmlr.org/format/format.html}{JMLR}
                \item \href{https://www.frontiersin.org/guidelines/author-guidelines}{Frontiers}
            \end{itemize}
            \vspace{0.05cm}
            \scriptsize \href{https://www.overleaf.com/gallery/tagged/academic-journal}{Overleaf Gallery}
        \end{column}
    \end{columns}
\end{frame}

% =====================================================================
% OPPSUMMERING
% =====================================================================
\section*{Oppsummering}

\begin{frame}{Oppsummering: Praktiske ferdigheter}
    \textbf{Jupyter og Python:}
    \begin{itemize}
        \item F01--F04: Colab, variabler, lister, funksjoner, pandas
    \end{itemize}
    
    \textbf{Maskinlæring:}
    \begin{itemize}
        \item F05--F06: scikit-learn workflow, matplotlib visualisering
    \end{itemize}
    
    \textbf{Spesialisert:}
    \begin{itemize}
        \item F07: NetworkX for nettverksanalyse
        \item F08: PyTorch for dyplæring
    \end{itemize}
    
    \textbf{Verktøy:}
    \begin{itemize}
        \item F09: AI-assistenter som kodehjelp
        \item F10: LaTeX/Overleaf for akademisk skriving
    \end{itemize}
    
    \vspace{0.3cm}
    \begin{block}{Lynkurs og Labs}
        Alle disse ferdighetene praktiseres gjennom Lynkurset og Lab 0--3. Øving gir mestring!
    \end{block}
\end{frame}

\end{document}

