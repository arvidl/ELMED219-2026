% =====================================================================
% ELMED219: Nevrosymbolsk AI og Agentisk AI
% Beamer-presentasjon - Momentliste S01-S11
% =====================================================================
\documentclass[aspectratio=169, 10pt]{beamer}

% =====================================================================
% PAKKER
% =====================================================================
\usepackage[utf8]{inputenc}
\usepackage[T1]{fontenc}
\usepackage[norsk]{babel}
\usepackage{graphicx}
\usepackage{tikz}
\usetikzlibrary{shapes.geometric, arrows, positioning, calc}
\usepackage{booktabs}
\usepackage{amsmath}
\usepackage{fontawesome5}

% =====================================================================
% TEMA OG FARGER
% =====================================================================
\usetheme{Madrid}
\usecolortheme{beaver}

% =====================================================================
% TITTELINFO
% =====================================================================
\title{Nevrosymbolsk AI og Agentisk AI}
\subtitle{ELMED219: Momentliste S01--S11}
\author{ELMED219}
\date{Vår 2026}

% =====================================================================
% DOKUMENT
% =====================================================================
\begin{document}

% Tittelside
\begin{frame}
    \titlepage
\end{frame}

% Innholdsfortegnelse
\begin{frame}{Oversikt}
    \tableofcontents
\end{frame}

% =====================================================================
% SEKSJON: Symbolsk vs. Konneksjonistisk AI
% =====================================================================
\section{Grunnleggende paradigmer}

\begin{frame}{S01: Kontrastere symbolsk og konneksjonistisk AI}
    \begin{columns}[T]
        \begin{column}{0.48\textwidth}
            \textbf{Symbolsk AI (GOFAI):}
            \begin{itemize}
                \item ``Good Old-Fashioned AI''
                \item Eksplisitte regler og symboler
                \item Logisk resonnering
                \item Kunnskapsrepresentasjon
                \item Ekspert-systemer
            \end{itemize}
            
            \vspace{0.3cm}
            \textbf{Styrker:}
            \begin{itemize}
                \item Forklarbar
                \item Presise resonnementer
                \item Kodifiserer ekspertkunnskap
            \end{itemize}
        \end{column}
        \begin{column}{0.48\textwidth}
            \textbf{Konneksjonistisk AI:}
            \begin{itemize}
                \item Nevrale nettverk, dyplæring
                \item Lærer fra data
                \item Subsymbolsk representasjon
                \item Mønstergjenkjenning
                \item ``Black box''
            \end{itemize}
            
            \vspace{0.3cm}
            \textbf{Styrker:}
            \begin{itemize}
                \item Lærer komplekse mønstre
                \item Håndterer støy og usikkerhet
                \item Skalerer med data
            \end{itemize}
        \end{column}
    \end{columns}
    
    \vspace{0.3cm}
    \begin{block}{Historisk perspektiv}
        AI har svingt mellom disse paradigmene. Nå: Kan vi kombinere dem?
    \end{block}
\end{frame}

\begin{frame}{S02: Forklare konseptet nevrosymbolsk integrasjon}
    \textbf{Nevrosymbolsk AI = det beste fra to verdener}
    
    \vspace{0.3cm}
    \textbf{Hovedidé:}
    \begin{itemize}
        \item Kombiner \textbf{nevralt} (læring fra data, mønstergjenkjenning)
        \item med \textbf{symbolsk} (resonnering, kunnskap, forklarbarhet)
    \end{itemize}
    
    \vspace{0.3cm}
    \textbf{Integrasjonsmønstre:}
    \begin{enumerate}
        \item \textbf{Neural $\rightarrow$ Symbolic:} Nevralt nettverk produserer symboler for resonnering
        \item \textbf{Symbolic $\rightarrow$ Neural:} Symbolsk kunnskap guider nevralt nettverk
        \item \textbf{Hybrid:} Tett integrasjon der begge informerer hverandre
    \end{enumerate}
    
    \vspace{0.3cm}
    \begin{block}{Eksempel: Medisinsk diagnose}
        \textbf{Nevral:} CNN analyserer MR-bilde, identifiserer tumor-features \\
        \textbf{Symbolsk:} Regelbasert klassifikasjon iht. WHO-kriterier \\
        \textbf{Integrert:} Kombinerer bildefunn med kliniske regler for diagnose
    \end{block}
\end{frame}

% =====================================================================
% SEKSJON: Kunnskapsgrafer
% =====================================================================
\section{Kunnskapsgrafer og ontologier}

\begin{frame}{S03: Beskrive hva en kunnskapsgraf er}
    \textbf{Kunnskapsgraf (Knowledge Graph):}
    \begin{itemize}
        \item Strukturert representasjon av kunnskap som en \textbf{graf}
        \item Noder = \textbf{entiteter} (personer, sykdommer, legemidler)
        \item Kanter = \textbf{relasjoner} (``behandles med'', ``forårsaker'')
    \end{itemize}
    
    \vspace{0.3cm}
    \textbf{Trippel-representasjon:}
    \begin{center}
        (Subjekt, Predikat, Objekt) \\
        \textit{(Diabetes, behandles\_med, Metformin)}
    \end{center}
    
    \vspace{0.3cm}
    \begin{columns}[T]
        \begin{column}{0.48\textwidth}
            \textbf{Anvendelser:}
            \begin{itemize}
                \item Google Knowledge Graph
                \item Medisinsk beslutningsstøtte
                \item Legemiddelinteraksjoner
            \end{itemize}
        \end{column}
        \begin{column}{0.48\textwidth}
            \textbf{Fordeler:}
            \begin{itemize}
                \item Maskinlesbar kunnskap
                \item Resonnering over relasjoner
                \item Koblet data (Linked Data)
            \end{itemize}
        \end{column}
    \end{columns}
    
    \begin{block}{Teknologier}
        RDF, OWL, SPARQL -- standarder for kunnskapsrepresentasjon og spørring
    \end{block}
\end{frame}

\begin{frame}{S04: Kjenne til medisinske ontologier}
    \textbf{Ontologi = formell spesifikasjon av begreper og relasjoner i et domene}
    
    \vspace{0.3cm}
    \textbf{Viktige medisinske ontologier/terminologier:}
    \begin{center}
        \footnotesize
        \begin{tabular}{lp{5cm}l}
            \toprule
            \textbf{Navn} & \textbf{Beskrivelse} & \textbf{Anvendelse} \\
            \midrule
            SNOMED CT & Kliniske termer, 350 000+ begreper & EPJ, klassifisering \\
            ICD-10/11 & Internasjonal sykdomsklassifikasjon & Rapportering, statistikk \\
            LOINC & Laboratorieundersøkelser & Labresultater \\
            RxNorm & Legemidler & Forskrivning \\
            \midrule
            WHO CNS 2021 & CNS-tumorklassifikasjon & Gliomdiagnose \\
            \bottomrule
        \end{tabular}
    \end{center}
    
    \vspace{0.3cm}
    \begin{block}{WHO 2021 CNS-klassifikasjon}
        Integrerer histologiske og molekylære kriterier. Eksempel: IDH-mutant astrocytom krever både histologi \textbf{og} IDH-mutasjonsstatus.
    \end{block}
\end{frame}

\begin{frame}{S05: Diskutere fordeler med nevrosymbolsk AI i medisin}
    \textbf{Potensielle fordeler:}
    
    \vspace{0.3cm}
    \begin{enumerate}
        \item \textbf{Forklarbarhet:} Symbolsk komponent gir tolkbare resonnementer
        
        \item \textbf{Dataholdighet:} Symbolsk kunnskap kompenserer for lite data
        
        \item \textbf{Robusthet:} Regelbaserte begrensninger forhindrer absurde prediksjoner
        
        \item \textbf{Oppdaterbarhet:} Ny medisinsk kunnskap legges til som regler/fakta
        
        \item \textbf{Samsvar med retningslinjer:} Kode WHO-kriterier, guidelines direkte
        
        \item \textbf{Validering:} Sjekk om prediksjoner er konsistente med kjent kunnskap
    \end{enumerate}
    
    \vspace{0.3cm}
    \begin{alertblock}{Hvorfor dette er viktig i medisin}
        Medisin er et domene der ekspertkunnskap er rik, forklarbarhet er kritisk, og feil kan være fatale. Ren dyplæring er ofte utilstrekkelig.
    \end{alertblock}
\end{frame}

% =====================================================================
% SEKSJON: Case: Gliomdiagnose
% =====================================================================
\section{Case: Nevrosymbolsk gliomdiagnose}

\begin{frame}{S06: Nevrosymbolsk AI for gliomdiagnostikk}
    \textbf{Case study fra Lab 3, Notebook 08:}
    
    \vspace{0.3cm}
    \textbf{Problemet:}
    \begin{itemize}
        \item Gliom-klassifikasjon krever både \textbf{bildeanalyse} (MRI) og \textbf{molekylære markører} (IDH, MGMT, 1p/19q)
        \item WHO 2021 krever integrasjon av flere informasjonskilder
    \end{itemize}
    
    \vspace{0.3cm}
    \textbf{Nevrosymbolsk løsning:}
    \begin{enumerate}
        \item \textbf{Nevral komponent:} CNN segmenterer tumor i MRI (BraTS)
        \item \textbf{Symbolsk komponent:} WHO-kriterier kodet som kunnskapsgraf
        \item \textbf{Integrasjon:} Regelbasert klassifikator bruker CNN-output + molekylære markører
    \end{enumerate}
    
    \vspace{0.3cm}
    \begin{block}{Resultat}
        Diagnose som er både \textbf{bildebasert} (CNN) og \textbf{kriteriekonform} (WHO) -- med forklarbar resonnering.
    \end{block}
\end{frame}

\begin{frame}{S07: Forklare hvordan kunnskapsgrafer kan validere nevrale prediksjoner}
    \textbf{Validering av CNN-prediksjoner med symbolsk kunnskap:}
    
    \vspace{0.3cm}
    \textbf{Scenario:}
    \begin{itemize}
        \item CNN predikerer: ``Glioblastom med IDH-mutasjon''
        \item Kunnskapsgraf inneholder: ``Glioblastom er per definisjon IDH-villtype (WHO 2021)''
        \item \textbf{Konflikt detektert!}
    \end{itemize}
    
    \vspace{0.3cm}
    \textbf{Valideringsmekanismer:}
    \begin{enumerate}
        \item \textbf{Konsistenssjekk:} Er prediksjonen logisk konsistent?
        \item \textbf{Regelvalidering:} Oppfyller prediksjonen nødvendige kriterier?
        \item \textbf{Plausibilitetssjekk:} Er kombinasjonen av funn realistisk?
    \end{enumerate}
    
    \vspace{0.3cm}
    \begin{alertblock}{Verdi}
        Fanger opp potensielle CNN-feil \textbf{før} de når kliniker. Øker tilliten til systemet.
    \end{alertblock}
\end{frame}

% =====================================================================
% SEKSJON: Agentisk AI
% =====================================================================
\section{Agentisk AI}

\begin{frame}{S08: Definere agentisk AI og dens kjerneegenskaper}
    \textbf{Agentisk AI = AI som handler autonomt mot mål}
    
    \vspace{0.3cm}
    \textbf{Kjerneegenskaper:}
    \begin{enumerate}
        \item \textbf{Autonomi:} Handler uten kontinuerlig menneskelig input
        \item \textbf{Målrettet:} Jobber mot definerte mål
        \item \textbf{Planlegging:} Bryter ned komplekse mål i deloppgaver
        \item \textbf{Verktøybruk:} Kan bruke eksterne verktøy (API, databaser)
        \item \textbf{Refleksjon:} Evaluerer egen fremgang, justerer strategi
        \item \textbf{Minne:} Husker kontekst over flere interaksjoner
    \end{enumerate}
    
    \vspace{0.3cm}
    \textbf{LLM-baserte agenter:}
    \begin{itemize}
        \item LLM som ``hjerne'' -- resonnerer og planlegger
        \item Verktøy-kall (function calling) for handlinger
        \item Eksempler: AutoGPT, LangChain agents, Claude tool use
    \end{itemize}
\end{frame}

\begin{frame}{S09: Beskrive hvordan en AI-agent kan orkestrere klinisk arbeidsflyt}
    \textbf{Eksempel: Agentisk AI for gliomutredning}
    
    \vspace{0.3cm}
    \textbf{Scenario:} Lege ber om ``Utred ny MR for mulig gliom''
    
    \vspace{0.3cm}
    \textbf{Agenten kan:}
    \begin{enumerate}
        \item \faSearch~\textbf{Hent data:} Koble til PACS, laste ned MR-bilder
        \item \faBrain~\textbf{Analyser:} Kjør BraTS-segmentering, beregn tumorvolum
        \item \faBookMedical~\textbf{Litteratur:} Søk PubMed for relevante studier
        \item \faDatabase~\textbf{Biobank:} Sjekk om molekylær analyse er tilgjengelig
        \item \faFile*~\textbf{Klassifiser:} Bruk WHO-kriterier for tentativ diagnose
        \item \faClipboard~\textbf{Rapporter:} Generer strukturert rapport for MDT-møte
    \end{enumerate}
    
    \vspace{0.3cm}
    \begin{block}{Orkestrering}
        Agenten \textbf{koordinerer} flere AI-systemer og datakilder mot ett klinisk mål -- uten at legen må gjøre hvert steg manuelt.
    \end{block}
\end{frame}

\begin{frame}{S10: Forklare konseptet human-in-the-loop i agentiske systemer}
    \textbf{HITL i agentisk AI:}
    
    \vspace{0.3cm}
    \begin{itemize}
        \item Selv autonome agenter trenger \textbf{menneskelig tilsyn}
        \item Spesielt viktig for beslutninger med store konsekvenser
    \end{itemize}
    
    \vspace{0.3cm}
    \textbf{Implementeringsmønstre:}
    \begin{enumerate}
        \item \textbf{Godkjenningspunkter:} Agent pauser for godkjenning før kritiske handlinger
        \item \textbf{Konfidensterskel:} Menneske involveres når agent er usikker
        \item \textbf{Audit trail:} Alle handlinger logges for review
        \item \textbf{Overriding:} Menneske kan avbryte eller korrigere agent
    \end{enumerate}
    
    \vspace{0.3cm}
    \begin{alertblock}{I medisinsk kontekst}
        Agent kan samle data og foreslå diagnose, men \textbf{legen} tar endelig beslutning. Agenten er en ``superkraftig assistent'', ikke en autonom beslutningstaker.
    \end{alertblock}
\end{frame}

\begin{frame}{S11: Diskutere etiske utfordringer med autonome AI-agenter i helsevesenet}
    \textbf{Etiske utfordringer:}
    
    \vspace{0.3cm}
    \begin{enumerate}
        \item \textbf{Ansvar:} Hvem er ansvarlig når en agent gjør feil?
        \begin{itemize}
            \item Utvikler? Sykehus? Lege som ``slapp løs'' agenten?
        \end{itemize}
        
        \item \textbf{Autonomi og kontroll:} Hvor mye autonomi er for mye?
        \begin{itemize}
            \item Risiko for utilsiktet handling
        \end{itemize}
        
        \item \textbf{Transparens:} Kan vi forstå hvorfor agenten tok beslutningene?
        
        \item \textbf{Sikkerhet:} Agenter med tilgang til systemer = angrepsflate
        
        \item \textbf{Dehumanisering:} Risiko for å redusere menneskelig kontakt i helse
        
        \item \textbf{Overtillit:} ``Automation bias'' -- blindt stole på agenten
    \end{enumerate}
    
    \vspace{0.3cm}
    \begin{block}{Prinsipper for ansvarlig agentisk AI}
        Gradvis autonomi, robust HITL, klar ansvarsfordeling, omfattende testing, kontinuerlig overvåking
    \end{block}
\end{frame}

% =====================================================================
% OPPSUMMERING
% =====================================================================
\section*{Oppsummering}

\begin{frame}{Oppsummering: Nevrosymbolsk AI og Agentisk AI}
    \textbf{Nevrosymbolsk AI:}
    \begin{itemize}
        \item S01--S02: Kombinerer nevralt (læring) og symbolsk (resonnering)
        \item S03--S04: Kunnskapsgrafer og medisinske ontologier
        \item S05--S07: Fordeler i medisin, case gliomdiagnose, validering
    \end{itemize}
    
    \vspace{0.3cm}
    \textbf{Agentisk AI:}
    \begin{itemize}
        \item S08: Autonome, målrettede AI-systemer med planlegging
        \item S09: Orkestrering av kliniske arbeidsflyter
        \item S10--S11: HITL og etiske utfordringer
    \end{itemize}
    
    \vspace{0.3cm}
    \begin{block}{Lab 3, Notebook 08}
        Utforsk en detaljert case study som demonstrerer nevrosymbolsk gliomklassifikasjon og agentisk orkestrering.
    \end{block}
\end{frame}

\end{document}
